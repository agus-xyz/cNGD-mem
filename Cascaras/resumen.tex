%---------------------------------------------------------------------
%
%                      resumen.tex
%
%---------------------------------------------------------------------
%
% Contiene el cap�tulo del resumen en ingl�s.
%
% Se crea como un cap�tulo sin numeraci�n.
%
%---------------------------------------------------------------------

\chapter{Abstract}
\cabeceraEspecial{Abstract}

Nowadays Wireless Sensor Networks are subject to development constraints 
and difficulties such as the increasing RF spectrum saturation.
This brings hindrances to Wireless Ad-hoc Sensor Networks deployment,
especially for critical and sensitive applications.

\vspace{.1cm}

Cognitive Networks (CN), leaning on a cooperative
communication model, represent a new paradigm aimed at
improving wireless communications. Cognitive Wireless Sensor
Networks (CWSNs) compound cognitive properties into common
WSNs, developing new strategies to mitigate difficulties arising
from the constraints these networks face regarding energy and
resources.

\vspace{.1cm}

It is important to investigate cognitive models to explore their
benefit over our WAHSNs. However, few platforms allow their
study due to their early research stage, and they still show scarce
or specific features. Investigations take place mainly over
simulators, which provide partial and incomplete results.

\vspace{.1cm}

This paper presents a versatile platform that brings together
cognitive properties into WSNs. It combines hardware and
software modules as an entire instrument to investigate CWSNs.
The hardware fits WSN requirements in terms of size, cost and
energy. It allows communication over three different RF bands,
becoming the only cognitive platform for WSNs with this capability.
Besides, its modular and scalable design is widely adaptable to
almost any WAHSN application.



\vspace{.5cm}

\begin{table}[h!]
\Large
\scalebox{0.8}{
\begin{tabular}{ l l }
\textbf{\emph{KEY WORDS}}:	& \emph{cognitive}, \emph{wireless sensor networks}, \emph{platform}.
\end{tabular}}
\end{table}

\endinput
% Variable local para emacs, para  que encuentre el fichero maestro de
% compilaci�n y funcionen mejor algunas teclas r�pidas de AucTeX
%%%
%%% Local Variables:
%%% mode: latex
%%% TeX-master: "../Tesis.tex"
%%% End:
