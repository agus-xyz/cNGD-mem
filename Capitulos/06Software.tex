%---------------------------------------------------------------------
%
%                          Cap�tulo 6
%
%---------------------------------------------------------------------

\chapter{Software}

\begin{FraseCelebre}
\begin{Frase}
Too long, can you feel it? \\
Too long, oh can you feel it ?
\end{Frase}
\begin{Fuente}
Thomas \& Guy-Manuel, Daft Punk
\end{Fuente}
\end{FraseCelebre}

\begin{resumen}
Resumen
\end{resumen}

%-------------------------------------------------------------------
\section{Tools}
%-------------------------------------------------------------------
\label{cap6:sec:tools}
%-------------------------------------------------------------------

%-------------------------------------------------------------------
\subsection{MPLAB X}
%-------------------------------------------------------------------
\label{cap4:sub:mplabx}
%-------------------------------------------------------------------

%-------------------------------------------------------------------
\subsection{Programmer: ICD 3}
%-------------------------------------------------------------------
\label{cap4:sub:icd3}
%-------------------------------------------------------------------

\figura{Bitmap/Capitulo4/icd3}{width=.8\textwidth}{fig:cap4:icd3}%
{ICD3 programmer configuration.}

\ac{ICD} 3 is a device belonging to Microchip products that allows to get the microcontroller programmed. In addition, it enables run-time debugging using the \ac{IDE}. Up to 6 breakpoints can be enabled. This feature is quite valuable regarding the platform has a development character. Further information can be consulted at the manual\cite{icd3}.

%-------------------------------------------------------------------
\section{Hardware Abstraction Layer}
%-------------------------------------------------------------------
\label{cap6:sec:hal}
%-------------------------------------------------------------------
%ALGUNOS CAMBIOS

%DESCRIBIR CAMBIOS


%-------------------------------------------------------------------
\section{Application Layer}
%-------------------------------------------------------------------
\label{cap6:sec:applicationLayer}
%-------------------------------------------------------------------

%-------------------------------------------------------------------
\subsection{Conscumption Application}
%-------------------------------------------------------------------

%-------------------------------------------------------------------
\subsection{Security Application}
%-------------------------------------------------------------------

%-------------------------------------------------------------------
\section{Test Benches}
%-------------------------------------------------------------------
\label{cap6:sec:testBenches}
%-------------------------------------------------------------------

% Variable local para emacs, para  que encuentre el fichero maestro de
% compilaci�n y funcionen mejor algunas teclas r�pidas de AucTeX
%%%
%%% Local Variables:
%%% mode: latex
%%% TeX-master: "../Tesis.tex"
%%% End:
