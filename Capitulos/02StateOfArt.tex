%---------------------------------------------------------------------
%
%                          Cap�tulo 2
%
%---------------------------------------------------------------------

\chapter{Cognitive Wireless Sensor Networks: State of Art}

\begin{FraseCelebre}
\begin{Frase}
There's not much I know about you\\
Fear will always make you blind\\
But the answer is in clear view\\
It's amazing what you'll find face to face
\end{Frase}
%\begin{Fuente}
%Fuente
%\end{Fuente}
\end{FraseCelebre}

\begin{resumen}
This chapter shows an approach to the fundamentals of Wireless Sensor Networks and
the current development state of new paradigms such as Cognitive Radio, Cognitive Networks or Cognitive 
Wireless Sensor Networks. It describes current related applications and implementations, and introduce terms used all along
this dissertation.
\end{resumen}


%-------------------------------------------------------------------
\section{Cognitive Radio and Cognitive Networks}
%-------------------------------------------------------------------
\label{cap2:sec:cognitiveRadio}
%-------------------------------------------------------------------

%-------------------------------------------------------------------
\subsection{Cognitive Radio}
%-------------------------------------------------------------------

Nowadays, it is widely accepted that the main limitation in next-generation wireless systems is bandwidth scarcity. 
It is commonly believed that there is a crisis of spectrum availability for wireless communications\cite{corvus}.
However, according to regulatory bodies as the \ac{FCC}\cite{fcc} or the \ac{OFCOM}, most radio frequency spectrum is under-utilized while some spectrum bands are heavily used. Military, amateur radio or satellital frecuencies, for instance, are insuffiently utilized compared to cellular networks or the overcrowded \ac{ISM} bands\cite{wpanIssues}.

Most of the spectrum is allocated to specific applications and the static assignment of the spectrum 
results in an inefficient use of it. Figure~\ref{fig:spectrum} shows how the actual utilization in the 3-4 GHz
frequency band is 0.5\% and drops to 0.3\% in the 4-5 GHz band. This seems totally in contradiction to the concern of spectrum
shortage.

Spectrum utilization depends strongly on time and place, however, fixed spectrum allocation prevents specific assigned frequencies 
from being used, even when this use would not cause noticeable interference to the assigned service. These facts lead to the current inneficiency situation where the utilization of the total spectrum can be considered around 10\% and more than 95\% of the use is below 3GHz.

%datos triples sacaos de http://www.cmpe.boun.edu.tr/WiCo/doku.php?id=research#cognitive_radio
%WiMax Networks & Cognitive Radio, namely WiCo, is a research group functioning under Computer Networks Research Laboratory (NetLab) of Computer Engineering Department at bla 

\figuraEx{Vectorial/Todo}{width=.5\textwidth}{fig:spectrum}%
{A snapshot of the spectrum utilization up to 6 GHz in an urban area: taken at
mid-day with 20 kHz resolution taken over a time span of 50 microseconds with a 30
degree directional antenna at Berkeley Wireless Research Center \cite{seung}.}{A snapshot of the spectrum utilization up to 6 GHz in an urban area at BWRC}

The concept of \ac{CR} was first published by Joseph Mitola III and Gerald Q. Maguire, Jr. in 1999\cite{mitola1} and later on within Mitola's phD Dissertation in 2000\cite{mitola2}. It describes a novel paradigm for wireless communication in which a wireless device changes its transmission or reception parameters to communicate efficiently. This alteration of parameters is based on the active monitoring of several factors in the external and internal radio environment, such as radio frequency spectrum, user behavior, and network state. The idea was thought of as an ideal goal towards which a \ac{SDR} platform should evolve.

The cognitive radio is defined as an intelligent wireless communication system that is aware of its environment and uses the methodology of under-
standing-by-building to learn from the environment and adapt to statistical variations in the input stimuli, with two primary objectives:
\begin{itemize}
\item Hihighly reliable communications.
\item Efficient utilization of the radio spectrum.
\end{itemize}

Although the concept of \ac{CR} was defined originally as an extension to \ac{SDR}\cite{mitola1}, which is able to reason about
external factors, recently the term is mostly used in a narrower sense. \ac{FCC} suggests in~\cite{fcc1} that any radio having the adaptive spectrum awareness should be referred to as \ac{CR}:
\begin{quote}

\emph{
``A cognitive radio (CR) is a radio that can change its transmitter parameters based on
interaction with the environment in which is operates. The majority of cognitive radios will
probably be SDRs (Software Defined Radios), but neither having software nor being field
programmable are requirements of a cognitive radio.'' 
}

\end{quote}
Attending to some subtle differences between systems we can differenciate two main different types of \ac{CR}:

\begin{itemize}
\item \emph{Spectrum-Sensing Cognitive Radio}, in which only the radio-frequency spectrum is considered.\cite{brainempower}
\item \emph{Full Cognitive Radio  or Mitola radio}, in which every possible parameter observable by a wireless node is considered.\cite{mitola1}
\end{itemize}    

Although cognitive radio was initially thought of as full cognitive radio, most research work focuses on spectrum-sensing cognitive radio,  particularly in the TV bands. The great problem in spectrum-sensing cognitive radio is designing high-quality spectrum-sensing devices and algorithms for exchanging the so called knoledge domain \cite{brainempower}. The practical implementation of spectrum-management functions is a complex and multifaceted issue, since it must address a variety of technical and legal requirements.

Picture \ref{CWSNsoftModel} ilustrates the changes the \ac{OSI} model suffer when affected by cognition. \ac{CR} model can be referred to the first and second layers, thus the \emph{Phisical} and \emph{Link} layers. 

\figura{Vectorial/Todo}{width=.5\textwidth}{fig:CWSNsoftModel}%
{CWSN protocol model}
%poner una unica pila


The main functions of \ac{CR} devices are:\cite{dsa}\cite{CRfunctionality}

\begin{itemize}

\item \emph{Spectrum sensing}: An important requirement is detecting unused spectrum and sharing it, without causing interferences to other users; Spectrum-sensing techniques may be grouped into three categories:

\begin{itemize}
\item \emph{Transmitter detection}: \ac{CR} must have the capability to determine if a signal from a primary transmitter is locally present in a certain spectrum. Enclosed here we can find approaches such as \emph{matched filter detection, energy detection or cyclostationary-feature detection} are common.

\item \emph{Cooperative detection}: Refers to spectrum-sensing methods where information from multiple \ac{CR} users is integrated\cite{sensing}.

\item \emph{Interference-based detection}. This technique is not so commonly used. 
\end{itemize}

\item \emph{Power Control}: Power control is used for both opportunistic spectrum access and spectrum sharing CR systems for finding the cut-off level in SNR supporting the channel allocation and imposing interference power constraints for the primary user's protection respectively. In~\cite{threshold} a joint power control and spectrum sensing is proposed for capacity maximization.

\item \emph{Spectrum management}: Capturing the best available spectrum to meet user communication requirements, while not creating undue interference to other users. \ac{CR} should decide on the best spectrum band (over the available range) to meet \ac{QOS} requirements; therefore, spectrum-management functions are required for \ac{CR}. Spectrum-management functions are classified as \emph{Spectrum analysis} and \emph{
Spectrum decision}.

\end{itemize}

Realizing that CR technology has the potential to exploit the inefficiently utilized licensed bands without causing interference to incumbent users, the FCC released a Notice of Proposed Rule Making which would allow unlicensed radios to operate in the TV-broadcast bands. The IEEE 802.22 working group, formed in November 2004, is tasked with defining the air-interface standard for wireless regional area networks (based on CR sensing) for the operation of unlicensed devices in the spectrum allocated to TV service \cite{80222}.

%-------------------------------------------------------------------
\subsection{Cognitive Networks}
%-------------------------------------------------------------------

On the last years, ``cognitive'' or ``smart'' have become \emph{trending topics} being applied to many fields, included
to communication technologies. Having a look into the 90s literature, easily at least we find mentions about
cognitive radios [1, 2] *********** REFS , smart radios [3] ************ REFS, smart antennas [4] ************ REFS,
cognitive packets [5] *********** REFS , smart packets [6] ************ REFS and cognitive networks (CNs) [7, 8] ******** REFS.
Nevertheless, there does not seem to exist a commonly accepted definition of what these terms mean when applied to 
networking technologies.

The concept of \ac{CN} has been hanging out the collective psyche of the networking and wireless researching field for long. 
The first approach was made by Mitola [1] when briefly describes how the \ac{CR} could interact within the system-level scope 
of a \ac{CN}. Saracco [9] talks about \ac{CN}s in his investigation into the future information technology. Mahonen et al. [7] discuss 
\ac{CN}s with respect to future mobile Internet Protocol (IP) networks. None of these previous references, however, express clearly what 
a \ac{CN} is, how it should work and which problems it should solve.

The role that \ac{CR} had in inspiring the formulation of \ac{CN} concept made, in some cases, \ac{CN}s being described as networks of 
\ac{CR}s ************** REFS[32,33]. Recent research can be divided into two categories: \ac{CRN}Cognitive Radio Networks (CRNs) and
CNs. 

For \ac{CRN}, Mitola mentions how \ac{CR}s could interact within the system-level scope of a CN [1]***. Neel [92]
and Haykin [2] continue this line of thinking, examinating multi-user networks of \ac{CR}s as a game.
The scope of \ac{CRN}s still remains primarily on \ac{MAC} and \ac{PHY} layers, but now operating with some end-to-end objective. 
In a \ac{CRN}, the individual radios take most of the cognitive decisions, although they may act in cooperation. Some suggested
applications for \ac{CRN}s include cooperative spectrum sensing [93, 94]********* and emergency radio
networks [95]**************. From a more general perspective, Raychaudhuri [96]************ presents an architecture
for \ac{CRN}s.

Regarding \ac{CN}s, Clark proposes, in which was perhaps the first mention of \ac{CN} rather than \ac{CRN},
a network that can

\begin{quotation}
``assemble itself given high level instructions, reassemble itself as requirements change, automatically discover when 
something goes wrong, and automatically fix a detected problem or explain why it cannot do so.''
\end{quotation}

This would be achieved with the use of a \ac{KP} Knowledge Plane (KP) that transcends layersand domains to make global 
cognitive decisions. The \ac{KP} will add intelligence and weight to the edges of the network, and context sensitivity 
to its core. Saracco postulated [9]**** that the change from network intelligence controlling resources to having context 
sensitivity will help "flatten" the network by moving network intelligence into the core and control further out to the 
edges of the network. CRNs differ from CNs in that their action space extends beyond the MAC and PHY layers and the network may 
consist of more than just wireless devices. Furthermore, CN nodes may be less autonomous than a CRN node.

EXAMPLE

3.1.2 A Simple Example
This example and its description are inspired and influenced by Daniel Friend's example in
our IEEE Communications Magazine article [101].
41

Figure 3.1: Simple relay network for a wireless network. Vertices represent wireless connec-
tivity.
To illustrate the need for end-to-end rather than link adaptations, consider an ad-hoc data
session between a source node
S
1
and a destination node
D
1
as shown in Figure 3.1. The
source node does not have enough power to reach
D
1
directly, so it must must route traffic
through intermediate nodes
R
1
and-or
R
2
. Assume that the end-to-end goal is to have the
highest probability of successful transmission. The routing layer will determine routes based
on minimum hop count which, in this case, includes either
R
1
or
R
2
. Node
S
1
will make a
link-layer adaptation, selecting between
R
1
and
R
2
based on their Signal to Interference and
Noise Ratio (SINR). From the standpoint of the link layer in node
S
1
, this ratio correlates
with the probability that the transmitted packets will arrive correctly at the relay node.
However, without additional information, this selection does not guarantee anything about
the end-to-end packet delivery probability from
S
1
to
D
1
.
In contrast to a link adaptation, the CN might use some combination of observations from
all nodes to compute the total path outage probabilities from
S
1
to
D
1
through
R
1
and
R
2
. This shows the benefit of an end-to-end scope, but there is another advantage to the
CN, its cognitive capability. To illustrate this, we modify the original scenario to include
both
S
1
and
S
2
as source nodes, both routing traffic through
R
2
. Suppose that the learning
mechanism measures outages by determining the fraction of packets successfully delivered
from the source to its destination
****************************************************************************

First definition of \ac{CN} was deployed by Thomas  *********** REF.

%its a network's ability to adapt to changes and in-
%teractions in the network, often resulting in sub-optimal performance. Limited in state,
%scope and response mechanisms, the network elements (consisting of nodes, protocol lay-
%ers, policies and behaviors) are unable to make intelligent adaptations. Communication of
%network state information is stifled by the layered protocol architecture, making individual
%elements unaware of the network conditions experienced by other elements. Any response
%that an element may make to network stimuli can only be made inside of its limited scope.
%The adaptations that are performed are typically reactive, taking place after a problem has
%occurred. In this dissertation, we advance the idea of
%cognitive networks
%, which have the
%promise to remove these limitations by allowing networks to observe, act, and learn in order
%to optimize their performance.

******************************* DE LA WEB *************************+
Dynamic Spectrum Access (DSA) technique aims to solve spectrum allocation problems. The overall system that learns the operating environment and adapts its operating parameters according to its surrounding and uses DSA technique for efficient spectrum usage is called Cognitive Radio Network (CRN). In the following figure, spectrum holes are employed by CRN in an example scenario. In different areas the usage of the spectrum differs, so spectrum hole locations and their durations vary. CRN uses these spectrum holes for providing service to its users without causing harm to other users. Therefore, the change of parameters are observed by active monitoring of several factors in the external and internal radio environment, such as radio frequency spectrum, user behavior and network state. 
***********************************************+
 
%-------------------------------------------------------------------
\section{Wireless Sensor Networks}
%-------------------------------------------------------------------
\label{cap2:sec:wirelessSensorNetworks}
%-------------------------------------------------------------------

A wireless sensor network (WSN) consists of spatially distributed autonomous sensors to monitor physical or environmental conditions, such as temperature, sound, pressure, etc. and to cooperatively pass their data through the network to a main location. The more modern networks are bi-directional, also enabling control of sensor activity. The development of wireless sensor networks was motivated by military applications such as battlefield surveillance; today such networks are used in many industrial and consumer applications, such as industrial process monitoring and control, machine health monitoring, and so on.

The WSN is built of ``nodes'' - from a few to several hundreds or even thousands, where each node is connected to one (or sometimes several) sensors. Each such sensor network node has typically several parts: a radio transceiver with an internal antenna or connection to an external antenna, a microcontroller, an electronic circuit for interfacing with the sensors and an energy source, usually a battery or an embedded form of energy harvesting. A sensor node might vary in size from that of a shoebox down to the size of a grain of dust, although functioning ``motes'' of genuine microscopic dimensions have yet to be created. The cost of sensor nodes is similarly variable, ranging from a few to hundreds of dollars, depending on the complexity of the individual sensor nodes. Size and cost constraints on sensor nodes result in corresponding constraints on resources such as energy, memory, computational speed and communications bandwidth. The topology of the WSNs can vary from a simple star network to an advanced multi-hop wireless mesh network. The propagation technique between the hops of the network can be routing or flooding.[1][2]

\figura{Vectorial/Todo}{width=.5\textwidth}{fig:WSNnodeModel}%
{WSN node model}

%-------------------------------------------------------------------
\section{Cognitive Wireless Sensor Networks}
%-------------------------------------------------------------------
\label{cap2:sec:cognitiveWirelessSensorNetworks}
%-------------------------------------------------------------------

%-------------------------------------------------------------------
\section{Current Devices}
%-------------------------------------------------------------------
\label{cap2:sec:currentDevices}
%-------------------------------------------------------------------


%-------------------------------------------------------------------
%\section*{\NotasBibliograficas}
%-------------------------------------------------------------------
%\TocNotasBibliograficas

%Citamos algo para que aparezca en la bibliograf�a\ldots
%\citep{ldesc2e}

%\medskip

%Y tambi�n ponemos el acr�nimo \ac{CVS} para que no cruja.

%Ten en cuenta que si no quieres acr�nimos (o no quieres que te falle la compilaci�n en ``release'' mientras no tengas ninguno) basta con que no definas la constante \verb+\acronimosEnRelease+ (en \texttt{config.tex}).


%-------------------------------------------------------------------
%\section*{\ProximoCapitulo}
%-------------------------------------------------------------------
%\TocProximoCapitulo

%...

% Variable local para emacs, para  que encuentre el fichero maestro de
% compilaci�n y funcionen mejor algunas teclas r�pidas de AucTeX
%%%
%%% Local Variables:
%%% mode: latex
%%% TeX-master: "../Tesis.tex"
%%% End:
