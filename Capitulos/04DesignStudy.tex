%---------------------------------------------------------------------
%
%                          Cap�tulo 4
%
%---------------------------------------------------------------------

\chapter{Design Study}

\begin{FraseCelebre}
\begin{Frase}
Surf it, scroll it, pause it, click it,\\
Cross it, crack it, switch - update it\\
Name it, rate it, tune it, print it,\\
Scan it, send it, fax - rename it\\
%Touch it, bring it, pay it, watch it
\end{Frase}
\begin{Fuente}
Thomas \& Guy-Manuel, Daft Punk
\end{Fuente}
\end{FraseCelebre}

\begin{resumen}
In this chapter the platform design is described. Design decisions, specifications, schemes and needed calculations will be exposed.
Detailled characterstics of any chosen component, including microcontroller, radio interfaces, power-supply options will be provided.
\end{resumen}



%-------------------------------------------------------------------
\section{Design decisions}
%-------------------------------------------------------------------
\label{cap4:sec:hardwareDescription}
%-------------------------------------------------------------------

Attending to the already described requirements and the conclusions obtained from the \ac{FCD} review, the following decisions have been taken before proceeding with the design process.

The device can access three different frequency bands over the spectrum, fulfilling one of the desired requirements described in section \ref{cap3:sec:systemRequirements}. The chosen operation frecuencies will be 434 MHz, 868 MHz and 2.4 GHz. Coupling the \ac{ISM} bands in most part of Europe. This bands are preferred for \ac{WSN} due to the data-rate, transmission power and transmission range parameters they offer. This configuration places the device on the bound of complexity and cost. At the same time, this feature is essential for a \ac{CWSN} proper investigation, giving radio communication opportunities not provided by any other similar device so far. Moreover, a full exploitation of the firwmware described in \ref{cap3:sec:halreview} cn be done.


%-------------------------------------------------------------------
\section{Altium Designer}
%-------------------------------------------------------------------
\label{cap4:sec:altiumDesigner}
%-------------------------------------------------------------------


Altium Designer is an Electronic Design Automation\footnote{Electronic design automation is a category of software tools for designing electronic systems such as printed circuit boards and integrated circuits.} software package for printed circuit board, \ac{FPGA} and embedded software design, and associated library and release management automation. It is developed and marketed by Altium Limited of Australia.

Schematic capture

Schematic capture module provides electronics circuit editing functionality, including:
%seleccionar caracteristicas
\begin{itemize}
      \item Component library management
      \item Integration with several component distributors allows search for components and access to manufacturer's data[2]
    \item   SPICE mixed-signal circuit simulation
      \item Netlist export
      \item Reporting and BoM facilities
       \item Multi-channel, hierarchical schematics and design re-use
\end{itemize}

PCB design

Printed circuit board design module of Altium designer allows:
\begin{itemize}
   \item Component footprint library management
   \item Component placement
   \item Manual trace routing, with support for differential pairs, multi-trace routing, pin-swapping and gate-swaping
   \item Automated multi-channel layout and routing
   \item Interactive 3D editing of the board and MCAD export to STEP
   \item Manufacturing files generation with support for Gerber and ODB++ formats
\end{itemize}


%porque altium. se usa en el labo. version
%-------------------------------------------------------------------
\subsection{agus lib}
%-------------------------------------------------------------------

%-------------------------------------------------------------------
\section{Main Board - cognitiveNextGenerationDevice}
%-------------------------------------------------------------------
\label{cap4:sec:cNGD}
%-------------------------------------------------------------------

%-------------------------------------------------------------------
\subsection{Description}
%-------------------------------------------------------------------

%-------------------------------------------------------------------
\subsection{Schematics}
%-------------------------------------------------------------------

%-------------------------------------------------------------------
\subsection{Power Supply System}
%-------------------------------------------------------------------

%-------------------------------------------------------------------
\subsubsection{Battery}
%-------------------------------------------------------------------

%-------------------------------------------------------------------
\subsection{Microcontroller}
%-------------------------------------------------------------------

%-------------------------------------------------------------------
\subsubsection{Pinout}
%-------------------------------------------------------------------

%-------------------------------------------------------------------
\subsection{Radio Interfaces}
%-------------------------------------------------------------------

%-------------------------------------------------------------------
\subsubsection{Software Controlled Power Supply}
%-------------------------------------------------------------------

%-------------------------------------------------------------------
\subsection{Timing}
%-------------------------------------------------------------------

%-------------------------------------------------------------------
\subsection{Expansion Headers}
%-------------------------------------------------------------------

%-------------------------------------------------------------------
\subsubsection{Pinout}
%-------------------------------------------------------------------

%-------------------------------------------------------------------
\subsubsection{Considerations}
%-------------------------------------------------------------------

%-------------------------------------------------------------------
\section{Transceivers - $\mu$Trans 434/868}
%-------------------------------------------------------------------
\label{cap4:sec:transceivers}
%-------------------------------------------------------------------

%-------------------------------------------------------------------
\subsection{Description}
%-------------------------------------------------------------------

%-------------------------------------------------------------------
\subsection{Schematics}
%-------------------------------------------------------------------

%-------------------------------------------------------------------
\section{Serial Communication Board - rs232SHIELD}
%-------------------------------------------------------------------
\label{cap4:sec:rs232shield}
%-------------------------------------------------------------------

%-------------------------------------------------------------------
\subsection{Description}
%-------------------------------------------------------------------
%-------------------------------------------------------------------
\subsection{Schematics}
%-------------------------------------------------------------------

%-------------------------------------------------------------------
\section{Transceivers - chargerSHIELD}
%-------------------------------------------------------------------
\label{cap4:sec:chargerShield}
%-------------------------------------------------------------------

%-------------------------------------------------------------------
\subsection{Description}
%-------------------------------------------------------------------
%-------------------------------------------------------------------
\subsection{Schematics}
%-------------------------------------------------------------------
%-------------------------------------------------------------------
\subsection{Simulation}
%-------------------------------------------------------------------

% Variable local para emacs, para  que encuentre el fichero maestro de
% compilaci�n y funcionen mejor algunas teclas r�pidas de AucTeX
%%%
%%% Local Variables:
%%% mode: latex
%%% TeX-master: "../Tesis.tex"
%%% End:
