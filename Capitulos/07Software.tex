%---------------------------------------------------------------------
%
%                          Cap�tulo 7
%
%---------------------------------------------------------------------

\chapter{Software}

\begin{FraseCelebre}
\begin{Frase}
Too long, can you feel it? \\
Too long, oh can you feel it ?
\end{Frase}
\begin{Fuente}
Thomas \& Guy-Manuel, Daft Punk
\end{Fuente}
\end{FraseCelebre}

\begin{resumen}
This chapter gives a vision about the main adaptations required for the firmware to be adapted to the platform, new software implementations and behaviour of the final demo application.
\end{resumen}

%-------------------------------------------------------------------
\section{Tools}
%-------------------------------------------------------------------
\label{cap7:sec:tools}
%-------------------------------------------------------------------

%-------------------------------------------------------------------
\subsection{MPLAB X}
%-------------------------------------------------------------------
\label{cap7:sub:mplabx}
%-------------------------------------------------------------------
MPLAB X \ac{IDE} is a multiplatform software program to develop applications for Microchip microcontrollers and digital signal controllers. It is called an Integrated Development Environment  because it provides a single integrated ``environment'' to develop code for embedded microcontrollers.

\figura{Bitmap/Capitulo7/mplabx}{width=.5\textwidth}{fig:cap7:mplabx}%
{MPLAB X IDE logo.}

MPLAB X is based on the open source NetBeans IDE from Oracle. Some of its main features are:

\begin{itemize}
    \item Supports Multiple Configurations within your projects.
    \item Support for multiple Debug Tools of the same type.
    \item Supports Live Parsing.
    \item Supports hyperlinks for fast navigation to declarations and includes.
    \item MPLAB X can Track Changes within your own system using local history.
\end{itemize}



%-------------------------------------------------------------------
\subsection{Programmer: ICD 3}
%-------------------------------------------------------------------
\label{cap7:sub:icd3}
%-------------------------------------------------------------------

\figura{Bitmap/Capitulo4/icd3}{width=.8\textwidth}{fig:cap4:icd3}%
{ICD3 programmer configuration.}

\ac{ICD} 3 is a device belonging to Microchip products that allows to get the microcontroller programmed. In addition, it enables run-time debugging using the \ac{IDE}. Up to 6 breakpoints can be enabled. This feature is quite valuable regarding the platform has a development character. Further information can be consulted at the manual\cite{icd3}.

%-------------------------------------------------------------------
\section{Hardware Abstraction Layer}
%-------------------------------------------------------------------
\label{cap7:sec:hal}
%-------------------------------------------------------------------
%DEPURACION

%ALGUNOS CAMBIOS Y DESCRIBIR POR ENCIMA

%-------------------------------------------------------------------
\subsection{Platform options}
%-------------------------------------------------------------------
\label{cap7:subsec:usbtraces}
%-------------------------------------------------------------------

%-------------------------------------------------------------------
\subsection{USB tracing}
%-------------------------------------------------------------------
\label{cap7:subsec:usbtraces}
%-------------------------------------------------------------------

%-------------------------------------------------------------------
\subsection{Radio Interfaces Power Control Functions}
%-------------------------------------------------------------------
\label{cap7:subsec:usbtraces}
%-------------------------------------------------------------------

%-------------------------------------------------------------------
\subsection{Headers and other new definitions}
%-------------------------------------------------------------------
\label{cap7:subsec:usbtraces}
%-------------------------------------------------------------------

%-------------------------------------------------------------------
\section{Demo Application Layer}
%-------------------------------------------------------------------
\label{cap7:sec:applicationLayer}
%-------------------------------------------------------------------



% Variable local para emacs, para  que encuentre el fichero maestro de
% compilaci�n y funcionen mejor algunas teclas r�pidas de AucTeX
%%%
%%% Local Variables:
%%% mode: latex
%%% TeX-master: "../Tesis.tex"
%%% End:
