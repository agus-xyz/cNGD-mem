%---------------------------------------------------------------------
%
%                          Cap�tulo 3
%
%---------------------------------------------------------------------

\chapter{Review Study}

\begin{FraseCelebre}
\begin{Frase}
Work It Harder Make It Better\\
Do It Faster, Makes Us stronger\\
More Than Ever Hour After\\
Our Work Is Never Over 
\end{Frase}
%\begin{Fuente}
%Fuente
%\end{Fuente}
\end{FraseCelebre}

\begin{resumen}
In this chapter it is defined a model for our cognitive node. Firstly, the First Cognitive Device is evaluated
to obtain some guidelines and potential improvements in our design. The final device implementation requirements and
constraints are defined. Finally, it is carried out a discussion and evaluation about different modules, exposing different 
commercial options and coming into final decissions.
\end{resumen}


%-------------------------------------------------------------------
\section{Cognitive Wireless Sensor Network Node Model}
%-------------------------------------------------------------------
\label{cap3:sec:nodeModel}
%-------------------------------------------------------------------

Basically, the \ac{CWSN} node scheme responds to a standard \ac{WSN} model based on a microcontroller,
radio and sensor interfacing, and power supply. However, a further review shows some general differences:

\begin{itemize}

\item \emph{Multiple Radio Interfacing}: Regarding \ac{CR} capabilities for spectrum management and \ac{DSA}, the node
	must include multiple radio interfaces enabling different spectrum bands.


\item \emph{Multiple Control/Sensor Interfacing}: Since the device will serve as a standard \ac{CWSN} platform, it must
	embbed different multi-purpose undefinned interfaces. ************* FOOTNOTE

\item \emph{Protocol Stacking}: In order to enable communication over the different interfaces the model,
	must provide to any included radio interface, proper access to a the lower layers of a communication protocol. ***** FOOTNOTE

\item \emph{Cognitive Layer}: To carry out all the cognition algorithms and computations.

\end{itemize}

Figure \ref{fig:CWSNnodeModel} gives a view about the basic model. It shows together software and hardware modules in a simplified way.
Additionally, cognition tasks compulsory require of several functions to take into account at the design stage: 




***** DE JUAN Y FER QUE TIENEN MUCHAS *************   

\figura{Vectorial/Todo}{width=.5\textwidth}{fig:CWSNnodeModel}%
{CWSN node model}
%poner una unica pila

\figura{Vectorial/Todo}{width=.5\textwidth}{fig:CWSNsoftModel}%
{CWSN protocol model}
%poner una unica pila


All the previous considerations, in addition to the requirements to be defined will come into the model of the 
final implementation.



%-------------------------------------------------------------------
\section{First Cognitive Device Evaluation}
%-------------------------------------------------------------------
\label{cap3:sec:fcdEvaluation}
%-------------------------------------------------------------------

%-------------------------------------------------------------------
\section{System Requirements}
%-------------------------------------------------------------------
\label{cap3:sec:systemRequirements}
%-------------------------------------------------------------------

%-------------------------------------------------------------------
\section{Technology}
%-------------------------------------------------------------------
\label{cap3:sec:technology}
%-------------------------------------------------------------------

%-------------------------------------------------------------------
\subsection{Microcontroller}
%-------------------------------------------------------------------

%-------------------------------------------------------------------
\subsection{Radio Interfaces}
%-------------------------------------------------------------------

%-------------------------------------------------------------------
\subsection{Serial Communication}
%-------------------------------------------------------------------

%-------------------------------------------------------------------
\subsection{Power Supply System}
%-------------------------------------------------------------------

%-------------------------------------------------------------------
\subsubsection{Consumption}
%-------------------------------------------------------------------

%-------------------------------------------------------------------
\subsection{Timing}
%-------------------------------------------------------------------

%-------------------------------------------------------------------
\subsection{Application Capabilities}
%-------------------------------------------------------------------

%-------------------------------------------------------------------
\section{Conclusions}
%-------------------------------------------------------------------
\label{cap3:sec:conclusions}
%-------------------------------------------------------------------

%-------------------------------------------------------------------
%\section*{\NotasBibliograficas}
%-------------------------------------------------------------------
%\TocNotasBibliograficas
%Citamos algo para que aparezca en la bibliograf�a\ldots
%\citep{ldesc2e}

%\medskip

%Y tambi�n ponemos el acr�nimo \ac{CVS} para que no cruja.

%Ten en cuenta que si no quieres acr�nimos (o no quieres que te falle la compilaci�n en ``release'' mientras no tengas ninguno) basta con que no definas la constante \verb+\acronimosEnRelease+ (en \texttt{config.tex}).


%-------------------------------------------------------------------
%\section*{\ProximoCapitulo}
%-------------------------------------------------------------------
%\TocProximoCapitulo

%...

% Variable local para emacs, para  que encuentre el fichero maestro de
% compilaci�n y funcionen mejor algunas teclas r�pidas de AucTeX
%%%
%%% Local Variables:
%%% mode: latex
%%% TeX-master: "../Tesis.tex"
%%% End:
