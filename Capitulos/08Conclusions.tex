%---------------------------------------------------------------------
%
%                          Cap�tulo 8
%
%---------------------------------------------------------------------

\chapter{Conclusions}

\begin{FraseCelebre}
\begin{Frase}
Frase
\end{Frase}
\begin{Fuente}
Fuente
\end{Fuente}
\end{FraseCelebre}


%-------------------------------------------------------------------
\section{Further Studies}
%-------------------------------------------------------------------
\label{cap8:sec:furtherStudies}
%-------------------------------------------------------------------

test bed

crear funciones en hal para perifericos en headers

trabajar con transceptores a fin de crear un unico transceptor con 2 bandas e incluso a�adir posibilidad de wake-on radio.

desarrollar shields de funcionalidades utiles tales como un acceso ethernet, 3g, gsm, otra de proposito general con puntos de soldadura, 

cambiar el conector programador PGE por una versi�n simplificada mas barata y que ocupe menos espacio, incluso a traves de los headers. Ademas seria imprescindible de cara a un testbed el desarrollo de un sistema programador sin cablear.

cambio de arquitectura -> cambio de firmware 

desarrollar firmware y protocolo desde 0

Over the air programming (OTAP)

wireless traces

% Variable local para emacs, para  que encuentre el fichero maestro de
% compilaci�n y funcionen mejor algunas teclas r�pidas de AucTeX
%%%
%%% Local Variables:
%%% mode: latex
%%% TeX-master: "../Tesis.tex"
%%% End:
