%---------------------------------------------------------------------
%
%                          Cap�tulo 1
%
%---------------------------------------------------------------------

\chapter{Introduction}

\begin{FraseCelebre}
\begin{Frase}
Something in the air
\end{Frase}
\begin{Fuente}
Thomas \& Guy-Manuel, Daft Punk
\end{Fuente}
\end{FraseCelebre}

\begin{resumen}
In this chapter, a global description of the project is shown. This global description covers an introduction to the context,
the need of this project and a brief description of it. It defines goals and the project organization in time. Besides, it is 
described the structure of this dissertation.  
\end{resumen}


%-------------------------------------------------------------------
\section{Background}
%-------------------------------------------------------------------
\label{cap1:sec:background}
%-------------------------------------------------------------------

A \ac{WSN} consists of a number of sensors spread across a geographical area. Each sensor has wireless communication capability and some level of intelligence for signal processing and networking of the data. \ac{WSN}s have recently emerged as a premier research topic since they provide a decentralized technological solution to address applications related to security and surveillance, industrial control, infrastructure maintenance, automatic environmental monitoring, domotics, localization and tracking or health. They have great long-term economic potential and ability to transform our lives.

Together with this significant growth, in the last years, it was a huge and yet increasing wireless networks and sevices deployment caused by the growth in the ``Internet of Things'' connectivity phenomenon\cite{internetofthings}, with wireless \ac{M2M} communications. Being wireless communications nowadays one of the hottest and fastest growing segment of electronics.

This increasing demand for wireless communication brought a remarkable saturation on certains bands of the radio spectrum. This fact is mainly due to a inneficcient assignment of the spectrum\cite{wpanIssues}. Some of the most heavily used bands are, for instance, unlicesed bands, free of cost, so-called \ac{ISM} bands.

\figura{Bitmap/Capitulo1/spectrum}{width=.7\textwidth}{fig:cap1:spectrum}%
{Electromagnetic frecuency spectrum. Radio spectrum covers under 300 GHz frequencies}

Regarding spectrum scarcity, most \ac{WSN} solutions operate on unlicensed frequency bands. Normally they use \ac{ISM} bands,
like the 2.4 GHz band, also used by Wi-Fi, Bluetooth or IEEE 802.15.4 devices. 

The spectrum saturation is reflected over the increasing interferences and other undesirable effects as noise, with a consistent impact over \ac{TX} power required, \ac{QOS}, etc. These troubles find their origin in the increasing amount of traffic, the heterogeneous networks and devices, as well as the inefficient use of the spectrum they do. Nowadays, techniques focused on expand spectrum efficiency over communications, such as those related to dynamic channel allocation, power control or link adaptation, seems totally exploited and new long-term solutions are required. 

To address this challenge, new techniques such as \ac{CR} arise. \ac{CR} was introduced by Mitola in 1999 \cite{mitola1}
and 2000\cite{mitola2} and it integrates into communication networks concepts as spectrum sensing, self-learning, self-management 
or context analysis. Basically, \ac{CR} enables opportunistic access to the spectrum through cooperation and context
awareness. Spectrum sensing capabilities and cooperation among devices allow for better spectrum utilization and \ac{QOS}.
The concept of \ac{CN} describes a \ac{WN} aware of its environment and able to adapt its communication parameters in order to achieve more reliable and efficient communications.

\ac{WSN}s are shown as one of the areas with the highest demand for cognitive networking. Their imposed power consumption constraints and hard operation conditions require an efficient operation mode, specially for their wireless communications, since these are normally the most enery-consumming task. However, there are also some challenges to face\cite{challenges} like opportunistic energy-efficient transport.

Despite the potential of \ac{CWSN}, they are not deeply explored yet. Real or simulated scenarios scarcely exist and 
realistic platforms are crucial to the improvement and development of this new field. The shortage and undevelopment of \ac{CWSN} devices
or test-beds contributes to the scarcity of results in this area. 

One of the current \ac{CWSN} node platform is the \ac{FCD}, developed in 2011\cite{fcd} by researchers at the laboratory where this project is framed, the \ac{LSI}. The \ac{LSI} belongs to the \ac{DIE} within the \ac{ETSIT} at \ac{UPM}. This laboratory hosts researching lines related to embedded systems, \ac{WSN}s and security. The implemented device become the first real generic platform to study \ac{CWSN}s hosting three radio transceivers, but it was just a first approach. The \ac{FCD} was still far to be a really valuable researching testbed platform. This platform did not fully satisfy requirements in terms of low power consumption, cost and size, and communication capabilities. 

Continuing with the development of the \ac{FCD}, some software implementation were carried out at \ac{LSI}. A combined protocol stack for
three different transceivers including a \ac{HAL}, described in \cite{juanpfc}, was created. This firmware offers
to the application layers ease of use facing communication taks and it brings memory and resources savings compared with
the previous three stacks based-on protocol scheme. As well, in order to deal with cognitive strategoies and agorithmics, a software module
that runs together with the application layer was launched. This module, called \ac{CRMODULE} implements the theoretical model described
in \cite{conbrok} and it is a crucial achievement for \ac{CWSN} development.  

This dissertation describes the development and operation of the \ac{CNGD}, a node platform for \ac{CWSN} development. It includes features and capabilities not found on current devices. \ac{CNGD} must enable and promote \ac{CWSN}s allowing test strategies and better investigation deployments. The final implementation must integrate the developed software modules \ac{HAL} and \ac{CRMODULE} together with a demo application layer that show the full operation of the device. 

%-------------------------------------------------------------------
\section{Goals}
%-------------------------------------------------------------------
\label{cap1:sec:goals}
%-------------------------------------------------------------------

The project main goal is to design and implement a node for a \ac{CWSN}. The implementation must meet, regarding its utility, strong efficiency, modularity and scalability constraints. A demo application layer, also to develop, must integrate the already implemented \ac{HAL} and \ac{CRMODULE}. 

\ac{CWSN} nodes are very limited devices in terms of memory, computational power or energy consumption. This fact is crucial and it must be taken into account, making the device valuable and affordable for researching purposes. 

The design must be modular, so it will not need to be redone entirely with every particular change, and scalable, allowing flexibility in the complexity of applications. On the other hand, as development platform, the software and hardware architecture must provide advantages when implementing cognitive strategies as a valuable tool (data extraction, data reliability, portability, versatility...).

It is important to make clear that the design must fill the gaps and deficiencies that current devices show, providing a wider radio access, powering control options or real \ac{WSN} features, for instance. Furthermore, a definition of new needs, together with an important review of the \ac{FCD} is needed. 

Here the main goal is fragmented into shorter subgoals:

\begin{itemize}

\item \emph{Analysis of previous conclusions and results}.
Study of already tested aspects. Definition of new tests 
to provide new data of interest. These data will help
to take decisions about transceivers, microcontroller,
power supply...

\item \emph{Requirements definition}. Study of needs and constraints 
(cost, resources, size, consumption...) according to the future running applications.
Definition of features and requirements. Definition of the demo application.

\item \emph{Hardware design}. Pursuit of deficiencies over 
the \ac{FCD} design and possible beneficial potentialities 
to achieve the new defined requirements. Design optimization.
Components selection and final layout development. 

\item \emph{Hardware debugging and implementation}.
Component disposition and soldering. PCB design validation.

\item \emph{Application layer development}.
Integration of the \ac{HAL} and \ac{CRMODULE} into the hardware.
Development of the demo application layer integrating
the two already named software modules enabling a 
full test of the platform. The application will include
tipical \ac{WSN} functionalities.

\item \emph{General purpose tests}. 
Tests oriented to prove the proper operation and verify
the achievement of the requirements and constraints compliance.

\item \emph{Results analysis and conclusions}.
Evaluation of capabilities. Study of weaknesses or 
posible improvements. Definition of further studies.

\end{itemize}

%-------------------------------------------------------------------
\section{Project Organization}
%-------------------------------------------------------------------
\label{cap1:sec:projectOrganization}
%-------------------------------------------------------------------

The organization this project has folled is here described:

\begin{itemize}

\item \emph{State of art review}.

At this first stage, \ac{CWSN}s related information and specific knowledge
was acquired. Also, it was an approach to the development tools. Goals:  

\begin{itemize}
\item \ac{WSN} and \ac{CR} state of art review. Review of the current commercial solutions.

\item Adaptation to the developement platform as well as other tools (workstation, repository, soldering station,
hardware at the lab...).

\item Analisys of related projects results, either completed or under development.
\end{itemize}

\item \emph{Hardware design and implementation}.

This phase covered from the first definitions and node specifications, to the system mounting and implementation. Goals:

\begin{itemize}
\item Requirements definition and general node specifications. Stablishment of constraints and design criteria.

\item Radio wireless interfaces and other used components evaluation (microcontroller, serial interfaces...).

\item Design optimizations respect to the \ac{FCD}.

\item Hardware full design. Selection of the components. Design adjustments and previous work verification. Layout design.

\item Hardware platform implementation.

\end{itemize}

\item \emph{Software design}

Once the hardware was operating, the work turned over the software. Goals:

\begin{itemize}

\item Integration of the developed \ac{HAL} and \ac{CRMODULE} into the hardware.

\item Software implementation of required functions and application. Source code debugging.

\item Generation of first documentation about the developed software modules.

\end{itemize}

\item \emph{Tests and evaluation}.

Finally the proper operation of the device was evaluated and its results analysed.

\begin{itemize}
\item Fully integrated hardware and software test. Spectrum sensing, power consumption,
autonomy and control, communication among nodes, connectivity to other devices, protocol stacks...
 
\item Results interpretation and conclusions review. Statement of further studies 
and future development lines. Found problems evaluation.

\end{itemize}

\item \emph{Documentation generation}.

Dissertation and other required documentation (wiki, papers, manuals...) writing. Review of the software documentation.

\end{itemize}
%-------------------------------------------------------------------
\section{Outline}
%-------------------------------------------------------------------
\label{cap1:sec:outline}
%-------------------------------------------------------------------
% Variable local para emacs, para  que encuentre el fichero maestro de
% compilaci�n y funcionen mejor algunas teclas r�pidas de AucTeX
%%%
%%% Local Variables:
%%% mode: latex
%%% TeX-master: "../Tesis.tex"
%%% End:
