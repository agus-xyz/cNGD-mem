%---------------------------------------------------------------------
%
%                          Cap�tulo 4
%
%---------------------------------------------------------------------

\chapter{Design Study}

\begin{FraseCelebre}
\begin{Frase}
Surf it, scroll it, pause it, click it,\\
cross it, crack it, switch - update it\\
name it, rate it, tune it, print it,\\
scan it, send it, fax - rename it\\
%Touch it, bring it, pay it, watch it
\end{Frase}
\begin{Fuente}
Thomas \& Guy-Manuel, Daft Punk
\end{Fuente}
\end{FraseCelebre}

\begin{resumen}
In this chapter the platform design is described. Design decisions, specifications, schemes and needed calculations will be exposed.
Detailled characterstics of any chosen component, including microcontroller, radio interfaces, power-supply options will be provided.
\end{resumen}



%-------------------------------------------------------------------
\section{Global Hardware Description}
%-------------------------------------------------------------------
\label{cap4:sec:hardwareDescription}
%-------------------------------------------------------------------

Attending to the already described requirements and the conclusions obtained from the \ac{FCD} review, the following decisions have been taken before proceeding with the design process.

The whole system is thought as teaser. Consisting on different fitteable modules giving flexibility to the scheme. Main module, \ac{CNGD}, is cornerstone. \ac{CNGD} suposes the main platform of the design, hosting the power supply and control system, possibities to embbed up to three \ac{RI}s, a control core unit, and other interfacing possibilites studied further on. As part of this modular design, the interfaces \ac{CNGD} hosts open possibilities to new extension board implementations through a pair of headers. These attachable boards, following the popular way these such of devices are called, will be referred as shields.  

Offering the chance of embbedding up to three \ac{RI}s, the device could access three different frequency bands over the spectrum, fulfilling one of the desired requirements described in section \ref{cap3:sec:systemRequirements}. The chosen operation frecuencies in our design are 434 MHz, 868 MHz and 2.4 GHz. Coupling the \ac{ISM} bands in most part of Europe. This bands are preferred for \ac{WSN} due to the data-rate, transmission power and transmission range parameters they offer. This configuration places the device on the bound of complexity and cost. At the same time, this feature is essential for a \ac{CWSN} proper investigation, giving radio communication opportunities not provided by any other similar device so far. Moreover, a full exploitation of the firwmware described in \ref{cap3:sec:halreview} can be done, since it offers an integrated MiWi$^{TM}$ stack for three \ac{RI}s.

The three \ac{RI} are based on the IEEE 802.15.4 standard for \ac{WPAN}s, commonly used in \ac{WSN}. Specifically, the set of interfaces will operate under MiWi$^{TM}$ or MiWi$^{TM}$P2P protocol. They both are proprietary protocols designed by Microchip Technology that uses small, low-power digital radios. It is open source option designed for low data transmission rates (up to 250Kbit/s) and short distance (up to 100 without obstacles), cost constrained networks. Main difference between MiWi$^{TM}$ and ZigBee$^{TM}$\footnote{A very popular wireless IEEE 802.15.4 compatible protocol employed at \ac{WSNs}.} is complexity. MiWi$^{TM}$ offers a much more simple operation, resulting in a lighter implementation. The size required for the MiWi$^{TM}$ stack is  3-10 KB on the ROM depending on the node role while ZigBee$^{TM}$ takes 20-40 KB. MiWi$^{TM}$ is free of cost and it does not require for licenses adquisition as long as Microchip components are used, in fact Microchip requires the use of MiWi$^{TM}$ for its products. The global design becomes simpler and more omogeneus sharing a common communication protocol for all the \ac{RI}s.

Since there were not found compatible and suitable size commercial options for 868 MHz and 434 MHZ \ac{RI}s, a custom full implementation of these interfaces was required. $\mu$Trans434/868 from now on. These \ac{RI}s are based on the MRF49XA Microchip transceiver and its full description is taken at section \ref{cap4:sec:transceivers}. Option at 2.4 GHz is a MRF24J40MA module.

For this work a pair of shields were designed. A simple serial communication complement that links one \ac{UART} and a classic rs232 interface over the so-called rs232SHIELD. This interface would facilitate the software development since the firmware just allowed debugging through the \ac{UART} at first and it is described in section \ref{cap4:sec:rs232shield}. A battery charger Ni-Mh and Ni-Cd was also designed and implemented as a utility for portability options. The device was called as chargerSHIELD. 

%-------------------------------------------------------------------
\section{Altium Designer}
%-------------------------------------------------------------------
\label{cap4:sec:altiumDesigner}
%-------------------------------------------------------------------

For the whole designing task, a \ac{EDA}\footnote{Electronic design automation is a category of software tools for designing electronic systems such as printed circuit boards and integrated circuits.} software will be use. Altium Designer, being one of the most important \ac{EDA} softwares, is commonly used by researchers at \ac{LSI}. All the \ac{CNGD} design process is carried through Altium Designer on its version 10. 

Altium Designer is software package for printed circuit board, \ac{FPGA} and embedded software design, and associated library and release management automation. It is developed and marketed by Altium Limited of Australia.

\figura{Bitmap/Capitulo4/altium}{width=.3\textwidth}{fig:cap4:altium}%
{Altium Designer logo}

Despite the amount of functionalities offered by Altium Designer the two main options needed for this project are schematic capture, for circuit edition, and \ac{PCB} design, for subsequent layout deployment.

Options provided by the schematic capture module, required for the design task, includes:
%DESARROLLAR CARACTERISTICAS **************************************************************************************
\begin{itemize}
	\item Component library management. 
	\item \ac{SPICE} mixed-signal circuit simulation.
      	\item Netlist export. 
     	\item Reporting and BoM facilities.
	\item Multi-channel, hierarchical schematics and design re-use.
\end{itemize}

On the other hand, required option from the \ac{PCB} design module are:
\begin{itemize}
	\item Component footprint library management.
   	\item Component placement.
   	\item Manual trace routing, with support for multi-trace routing. 
   	\item Interactive 3D editing of the board and MCAD export to STEP
   	\item Manufacturing files generation with support for Gerber formats
\end{itemize}

%-------------------------------------------------------------------
\subsection{agus lib}
%-------------------------------------------------------------------
For the accomplish the full design of the platform over Altium Designer, a component library with all the requried but not found models was set all along the design proccess.

As said, the library includes models from all those components needed but not found as well as modifications of existing models in order tu suit the design criteria. Every contained model embbeds schematic, footprint and 3D model. 3D models are useful for 3D rendering and proper hardware fitting checkings purposes.

ANEXAR �libreria? **********************************************


%-------------------------------------------------------------------
\section{Transceivers - $\mu$Trans 434/868}
%-------------------------------------------------------------------
\label{cap4:sec:transceivers}
%-------------------------------------------------------------------

The chosen firmware for our platform gives support for MRF49XA modules over 434 MHz and 868 MHz bands. Specifically, the modules used during the firmware development were the MRF49XA PICtail$^{TM}$ Daughter Board**********REF, whose picture is seen in. This board is a demonstration and development daughter board for the MRF49XA ISM Band Sub-GHz RF Transceiver. The board can plug into a fitteable header, which makes it unsuitable for a reduced size and robust \ac{WSN} design. It either accepts connection to external antennas, which is a desirable requirement.

\figura{Bitmap/Capitulo4/pictail}{width=.4\textwidth}{fig:cap4:pictail}%
{MRF49XA PICtail$^{TM}$ Daughter Board}

No commercial \ac{RI}s were found for based on the MRF49XA transceiver showing a reduced size. The implementation of a MRF49XA-based \ac{RI} module, suitable on size and features for our desing was required.  

Previous to the \ac{CNGD} designing, which will embbeds the $\mu$Trans 434/868, the design of these \ac{RI}s was needed.

%-------------------------------------------------------------------
\subsection{Description}
%-------------------------------------------------------------------
MRF49XA is a low-power fully integrated Sub-GHz RF transceiver using FSK baseband modulation. An ideal choice for low-cost, high-volume, low data rate (< 256 kbps), two-way, short range wireless applications. It can operate in the unlicensed 433, 868 and 915 MHz frequency bands. The transceiver is integrated with different Sleep modes and an internal wake-up timer to reduce the overall current consumption. The device operates in the low-voltage range of 2.2V to 3.8V, and in Sleep mode, it operates at a very low-current state, typically 0.3 $\mu$A.

Further details about consumption on different modes and other features are given at table \ref{}.
\begin{itemize}
	\item Frequency bands of operation:
	\item Temperature range of operation:	
	\item RF dynamic range:
	\item Data rate:
	\item Power consumption: 
\end{itemize}

This transceiver is not IEEE 802.15.4 compatible since the number of channels it provides depend on the configured bit-rate, whereas the standard sets a fix number of channels. 

\figura{Bitmap/Capitulo4/mrf49xapinout}{width=.6\textwidth}{fig:cap4:mrf49xapinout}%
{MRF49XA pin diagram}

Figure \ref{fig:cap4:mrf49xapinout} shows the MRF49XA pin diagram, a full description of each pin is given here: 

\begin{itemize}
	\item 1. \emph{SDI}. Digital Input. Serial data input interface to MRF49XA (\ac{SPI} input signal).
	\item 2. \emph{SCK}. Digital Input. Serial clock interface (\ac{SPI} clock).
	\item 3. \emph{nCS}. Digital Input. Serial interface chip select (\ac{SPI} chip/device select).
	\item 4. \emph{SDO}. Digital Output. Serial data output interface from MRF49XA (\ac{SPI} output signal).
	\item 5. \emph{IRO}. Digital Output. Interrupt Request Output: Receiver generates an active-low interrupt request for the 					microcontroller on the determinated events. Some of this events are:
	\begin{itemize}
	\item Dataflow control registers notifications. 
	\item Negative pulse on interrupt input pin.
	\item Wake-up timer time-out.
	\item Supply voltage below the preprogrammed value is detected.
	\item Power-on Reset.
	\end{itemize}

	\item 6. \emph{FSK/DATA/FSEL}. Digital Input/Output. Frequency Shift Keying: Transmit FSK data input (with internal pull-up resistor of 133 k$\Omega$).
Data: When configured as DATA, this pin functions as follows:
- Data In: Manually modulates the data from the external host microcontroller when the internal TXBREG is disabled. If the TXBREG is enabled, this pin can be tied ``high'' or left unconnected. When reading the internal RXFIFOREG, this pin must be pulled ``low''.
- Data Out: Receives data in conjunction with RCLKOUT when the internal FIFO is not used.
FIFO Select: Selects the FIFO and the first bit appears on the next clock when reading the RXFIFOREG. The FSEL pin has an internal pull-up resistor. This pin must be ``high'' when the TX register is enabled. In order to achieve minimum current consumption, keep this pin ``high'' in Sleep mode.

	\item 7. \emph{RCLKOUT/FCAP/FINT}. Digital Input/Output Recovery Clock Output:
Provides the clock recovered from
the incoming data if:

- FTYPE bit of BBFCREG (see Ta b l e 2 - 1 0 ) is configured as digital filter and
- FIFO is disabled by configuring FIFOEN bit of GENCREG (see Ta b l e 2 - 1 0)

Filter Capacitor:
This pin is a raw baseband data if the
FTYPE bit of BBFCREG is conf
igured as a configuration
filter. The pin can be used by the host microcontroller for data
recovery.

FIFO Interrupt:
When the internal FIFO, FIFOEN bit of
GENCREG is enabled, this pin acts as a FIFO full interrupt,
indicating that the FIFO has be
en filled to its preprogrammed
limit (see FFBC<3:0> bits in FIFORSTREG in Ta b l e 2 - 1 0)

	\item 8. \emph{CLKOUT} Digital Output Clock Output: The transceiver's clock output can be used by the host microcontroller as a clock source. Refer Register 2 for more details

	\item 9. \emph{RFXTL/EXTREF} Analog Input RF Crystal:
This pin is connected to a 10 MHz series crystal
or to an external oscillator reference. The crystal is used as a
reference for the PLL which generates the local oscillator
frequency. It is possible to ``pull'' the crystal to the accurate
frequency by changing the load capacitor value.
External Reference Input:
An external reference input, such
as an oscillator, can be connected as a reference source.
Connect the oscillator through a 0.01 $\mu$F capacitor

	\item 10. \emph{nRESET} Digital Input/Output Active-
low hardware pin. This pin has an open-drain Reset
output with internal pull-up and input buffer. Refer to
Section 3.1, Reset
for more details

	\item 11. \emph{Vss} Ground Ground reference

	\item 12. \emph{RFP} RF Input/Output Differential RF input/output (+)

	\item 13. \emph{RFN} RF Input/Output Differential RF input/output (-)

	\item 14. \emph{VDD} Power RF power supply. Bypass with a capacitor close to the pin. See Section 2.1, Power and Ground Pins
for more details.

	\item 15. \emph{RSSIO} Analog Input/Output Received Signal Strength. Indicator Output:
The analog
RSSI output is used to determi
ne the signal strength. The
response and settling time depends on the external filter
capacitor. Typically, a 4-10 nF capacitor provides optimum
response time for most applications.

	\item 16. \emph{nINT/DIO} Digital Input/Output Interrupt:This pin can be configured as an active-low external interrupt to the device. If a logic '0' is applied to this
pin, it causes the IRO
pin to toggle, signaling an interrupt to
the external microcontroller. Th
e source of interrupt can be
determined by reading the first four bits of STSREG (see Ta b l e 2 - 4). This pin can be used
to wake-up the device from Sleep.
Data Indicator Output:
This pin can be configured to
indicate valid data based on the actual internal settings.
\end{itemize}

The transmitter with a direct conversion architecture has a typical output power of +7 dBm. An internal transmit/receive switch combines the transmitter and receiver circuits into differential RFP and RFN pins. These pins are connected to the impedance matching circuitry (Balun) and to the external antenna connected to the device.

The quality of the data is checked or validated using the RSSI and DQI blocks built into the transceiver. Data is buffered in transmitter registers and receiver FIFOs. The Automatic Frequency Control feature allows the use of a low-accuracy and low-cost crystal. The CLKOUT is used
to clock the external controller. The transceiver is controlled via a 4-wire SPI, interrupt (INT/DIO and IRO), FSK/DATA/FSEL, RCLKOUT/FCAP/FINT and RESET pins. The interface between the microcontroller and MRF49XA is shown in Figure \ref{fig:cap4:mrf49xainterface}.


\figura{Bitmap/Capitulo4/mrf49xainterface}{width=0.5\textwidth}{fig:cap4:mrf49xainterface}%
{Microcontroller to MRF49XA interface}

The complete design of the $\mu$Trans 434/868 is based on the implementation guidelines provided my Microchip at MRF49XA datasheet. The only difference between 434 Mhz and 868 Mhz version of the $\mu$Trans is the impedance matching circuitry facing the external antenna. The components composing it are different for each frequency band, values are specified at table ************************ \ref{fig:cap4:transpurposedschematic}. 


%-------------------------------------------------------------------
\subsection{Schematics}
%-------------------------------------------------------------------

\figura{Bitmap/Capitulo4/mtransschematic}{width=.8\textwidth}{fig:cap4:mtransschematic}%
{$\mu$Trans 434/868 schematic}

% TABLA CON VALORES
 
\figura{Bitmap/Capitulo4/mtransmodel}{width=.4\textwidth}{fig:cap4:mtransmodel}%
{$\mu$Trans 434/868 component model for use}

%-------------------------------------------------------------------
\section{Main Board - cognitiveNextGenerationDevice}
%-------------------------------------------------------------------
\label{cap4:sec:cNGD}
%-------------------------------------------------------------------

\figura{Bitmap/Capitulo4/modelcngd}{width=.7\textwidth}{fig:cap4:modelcngd}%
{Architecture Model of the cNGD}

%-------------------------------------------------------------------
\subsection{Description}
%-------------------------------------------------------------------

As part of the main goal, the firmware reviewed in section \ref{cap3:sec:halreview} will be used. It supposes an aproach to the requirements achievement since it gives support for three interfaces, it supposes a reduction on the computation load and therefore, on the needed resources and power consumption. Moreover it will facilitate the development, offering a stable firmware version to work over and avoiding software development tasks. This fact will force the design to keep one architecture compatible with the firmware, hence a 32-bit Microchip PIC is required as microcontroller.

Core unit is a PIC32675F256L, a midrange microcontroller from the mentioned family. Being the humblest of the options meeting memory performance and peripherals requirements. On the other hand, its 100 pins allow a better access of the peripherals provided.


The device offers serial communication through the already named $\mu$USB options. Moreover, it gives access to to different peripherals through a pair of 20-pin headers. The list of approachable peripherals and pins covers battery connection (for charging purposes), \ac{GPIO}s, \ac{MCLR} pin, external interruptions, analogue inputs, \ac{USB}, Ethernet module, I2C bus, \ac{UART}s and \ac{SPI}. These option gives flexibility, modularity and versatility to the device. It provides possibilities for multiple kind of applications and open the design to new implementations and extensions for a particular aplications.

Power supply will be able to take place through three ways, a $\mu$USB, a block terminal and a DC terminal. Two first options will take a 5V supply, whereas the third one will take 3.3V. This last option is thought to be supplied by a battery. $\mu$USB option serves as serial communication chance, and since \ac{USB} provides 5V power supply, this is employed. A second 5V supplying connector opens possibilites to not compulsory use\ac{USB}. A software-driven power supply system to the \ac{RI}s allows the application to switch the power supply to these modules. 
   
%-------------------------------------------------------------------
\subsection{Schematics}
%-------------------------------------------------------------------

\figura{Bitmap/Capitulo4/schematic}{width=1.1\textwidth}{fig:cap4:gneralschematic}%
{cNGD general schematic}

\figura{Bitmap/Capitulo4/MCU}{width=1.1\textwidth}{fig:cap4:mcuschematics}%
{Microcontroller specific schematic}

\figura{Bitmap/Capitulo4/powersupply}{width=1.1\textwidth}{fig:cap4:powersupplyschematic}%
{Power supply system schematic}

\figura{Bitmap/Capitulo4/headers}{width=1.1\textwidth}{fig:cap4:headerschematic}%
{Expansion header system schematic}

%-------------------------------------------------------------------
\subsection{Power Supply System}
%-------------------------------------------------------------------



%-------------------------------------------------------------------
\subsubsection{Battery}
%-------------------------------------------------------------------

%-------------------------------------------------------------------
\subsection{Microcontroller}
%-------------------------------------------------------------------


%-------------------------------------------------------------------
\subsubsection{Pinout}
%-------------------------------------------------------------------

\begin{table}[htbp]
\begin{tabular}{|l|l|l|l|l|}
\hline
PIN & TIPO & TIPO 2PERIFERICO & USO \\ \hline
1 & RE5 & MiWi & CS* \\ \hline
2 & RE6 & MiWi & Wake \\ \hline
3 & RE7 & MiWi & RST* \\ \hline
4 & SCK2A & U2BTX/RG6WIFI & sck2 \\ \hline
5 & SDI2A & U2ARX/RG7WIFI & sdo2 \\ \hline
6 & SDO2A & U2ATX/RG8WIFI & sdi2 \\ \hline
7 & MCLR* &  & RESET \\ \hline
8 & SS2A* & U2BRX/RG9 &  \\ \hline
9 & VSS &  & TIERRA \\ \hline
10 & VDD &  & ALIMENTACI�N \\ \hline
11 & Vbuson & RB5USB otg & Control Bomba de Carga \\ \hline
12 & RB4 &  & LED \\ \hline
13 & RB3 &  & LED \\ \hline
14 & RB2 &  & LED \\ \hline
15 & PGEC1 & RB1 & Programador \\ \hline
16 & PGED1 & RB0 & Programador \\ \hline
17 & RB6/AN6 & PGEC2 & libre \\ \hline
18 & RB7/AN7 & PGED2 & libre \\ \hline
19 & AVDD &  & ALIMENTACI�N \\ \hline
20 & AVSS &  & TIERRA \\ \hline
21 & RB8/AN8 & SS3A* & libre \\ \hline
22 & RB9/AN9 &  & libre \\ \hline
23 & TMS & RB10 & jtag \\ \hline
24 & TDO & RB11 & jtag \\ \hline
25 & VSS &  & TIERRA \\ \hline
26 & VDD &  & ALIMENTACI�N \\ \hline
27 & TCK & RB12 & jtag \\ \hline
28 & TDI & RB13 & jtag \\ \hline
29 & SCK3A & RB14MiWi & sck3 \\ \hline
30 & RB15 & TULIO & RST* \\ \hline
31 & SDI3A & RF4MiWi & sdo3 \\ \hline
32 & SDO3A & RF5MiWi & sdi3 \\ \hline
33 & USBID & RF3USB otg &  \\ \hline
34 & Vbus & USB otg &  \\ \hline
35 & Vusb & USB otg & Conexi�n a VCC con Condensador \\ \hline
36 & D- & RG3USB otg &  \\ \hline
37 & D+ & RG2USB otg &  \\ \hline
38 & VDD &  & ALIMENTACI�N \\ \hline
39 & OSC1 & RC12 & Cristal de cuarzo 8mhz \\ \hline
40 & OSC2 & RC15 & Cristal de cuarzo 8mhz \\ \hline
41 & VSS &  & TIERRA \\ \hline
42 & INT1 & RD8/RTCCWiFi & int* \\ \hline
43 & U1BRX & INT2/SS1A*/RD9RS-232 & UART1B RS-232 RX \\ \hline
44 & INT3 & RD10MiWi & int \\ \hline
45 & RD11 & INT4TULIO & interrupci�n \\ \hline
46 & INT0 & RD0 &  \\ \hline
47 & SOSCI & RC13 & Cristal de cuarzo 32kHz \\ \hline
48 & SOSCO & RC14/T1CK & Cristal de cuarzo 32kHz \\ \hline
49 & U1BTX & RD1/SCK1ARS-232 & UART1B RS-232 TX \\ \hline
50 & U1ARX & RD2/SDI1ATULIO & UART1A RX \\ \hline
51 & U1ATX & RD3/SDO1ATULIO & UART1A TX \\ \hline
52 & RD4 &  & PULSADOR \\ \hline
53 & RD5 &  & PULSADOR \\ \hline
54 & RD6 &  &  \\ \hline
55 & RD7 &  &  \\ \hline
56 & VDDcore &  & VCC, con dos condensadores \\ \hline
57 & VDD &  & ALIMENTACI�N \\ \hline
58 & RF0 &  &  \\ \hline
59 & RF1 &  &  \\ \hline
60 & RE0 &  & LED \\ \hline
61 & RE1 & WIFI & CS* \\ \hline
62 & RE2 & WIFI & Hibernate \\ \hline
63 & RE3 & WIFI & WP* \\ \hline
64 & RE4 & WIFI & RST* \\ \hline
\end{tabular}
\caption{a
\label{a}}
\end{table}


%-------------------------------------------------------------------
\subsection{Radio Interfaces}
%-------------------------------------------------------------------

\figura{Bitmap/Capitulo4/434}{width=0.7\textwidth}{fig:cap4:434schematics}%
{Radio Interface system schematic}


The MiWi P2P protocol stack supports star and peer-to-peer wireless-network topologies, useful for simple, short-range, wireless node-to-node communication. Additionally, the stack provides sleeping-node, active-scan and energy-detect features while supporting the low-power requirements of battery-operated devices.


\figura{Bitmap/Capitulo4/miwistack}{width=.4\textwidth}{fig:cap4:miwistack}%
{MiWi$^{TM}$ Protocol stack.}
%-------------------------------------------------------------------
\subsubsection{Software Controlled Power Supply}
%-------------------------------------------------------------------

%-------------------------------------------------------------------
\subsection{Timing}
%-------------------------------------------------------------------

%-------------------------------------------------------------------
\subsection{Expansion Headers}
%-------------------------------------------------------------------




%-------------------------------------------------------------------
\subsubsection{Pinout}
%-------------------------------------------------------------------

%-------------------------------------------------------------------
\subsubsection{Considerations}
%-------------------------------------------------------------------

%-------------------------------------------------------------------
\section{Serial Communication Board - rs232SHIELD}
%-------------------------------------------------------------------
\label{cap4:sec:rs232shield}
%-------------------------------------------------------------------

%-------------------------------------------------------------------
\subsection{Description}
%-------------------------------------------------------------------
%-------------------------------------------------------------------
\subsection{Schematics}
%-------------------------------------------------------------------

%-------------------------------------------------------------------
\section{Transceivers - chargerSHIELD}
%-------------------------------------------------------------------
\label{cap4:sec:chargerShield}
%-------------------------------------------------------------------

%-------------------------------------------------------------------
\subsection{Description}
%-------------------------------------------------------------------
%-------------------------------------------------------------------
\subsection{Schematics}
%-------------------------------------------------------------------
%-------------------------------------------------------------------
\subsection{Simulation}
%-------------------------------------------------------------------

% Variable local para emacs, para  que encuentre el fichero maestro de
% compilaci�n y funcionen mejor algunas teclas r�pidas de AucTeX
%%%
%%% Local Variables:
%%% mode: latex
%%% TeX-master: "../Tesis.tex"
%%% End:
