%---------------------------------------------------------------------
%
%                          Cap�tulo 2
%
%---------------------------------------------------------------------

\chapter{Cognitive Wireless Sensor Networks: State of Art}

\begin{FraseCelebre}
\begin{Frase}
There's not much I know about you\\
Fear will always make you blind\\
But the answer is in clear view\\
It's amazing what you'll find face to face
\end{Frase}
%\begin{Fuente}
%Fuente
%\end{Fuente}
\end{FraseCelebre}

\begin{resumen}
This chapter shows an approach to the fundamentals of Wireless Sensor Networks and
the current development state of new paradigms such as Cognitive Radio, Cognitive Networks or Cognitive 
Wireless Sensor Networks. It describes current related implementations and introduce terms used all along
this dissertation.
\end{resumen}


%-------------------------------------------------------------------
\section{Cognitive Radio}
%-------------------------------------------------------------------
\label{cap2:sec:cognitiveRadio}
%-------------------------------------------------------------------

%-------------------------------------------------------------------
\section{Cognitive Networks}
%-------------------------------------------------------------------
\label{cap2:sec:cognitiveNetworks}
%-------------------------------------------------------------------

%-------------------------------------------------------------------
\section{Wireless Sensor Networks}
%-------------------------------------------------------------------
\label{cap2:sec:wirelessSensorNetworks}
%-------------------------------------------------------------------

%-------------------------------------------------------------------
\section{Cognitive Wireless Sensor Networks}
%-------------------------------------------------------------------
\label{cap2:sec:cognitiveWirelessSensorNetworks}
%-------------------------------------------------------------------

%-------------------------------------------------------------------
\section{Current Devices}
%-------------------------------------------------------------------
\label{cap2:sec:currentDevices}
%-------------------------------------------------------------------


%-------------------------------------------------------------------
%\section*{\NotasBibliograficas}
%-------------------------------------------------------------------
%\TocNotasBibliograficas

%Citamos algo para que aparezca en la bibliograf�a\ldots
%\citep{ldesc2e}

%\medskip

%Y tambi�n ponemos el acr�nimo \ac{CVS} para que no cruja.

%Ten en cuenta que si no quieres acr�nimos (o no quieres que te falle la compilaci�n en ``release'' mientras no tengas ninguno) basta con que no definas la constante \verb+\acronimosEnRelease+ (en \texttt{config.tex}).


%-------------------------------------------------------------------
%\section*{\ProximoCapitulo}
%-------------------------------------------------------------------
%\TocProximoCapitulo

%...

% Variable local para emacs, para  que encuentre el fichero maestro de
% compilaci�n y funcionen mejor algunas teclas r�pidas de AucTeX
%%%
%%% Local Variables:
%%% mode: latex
%%% TeX-master: "../Tesis.tex"
%%% End:
