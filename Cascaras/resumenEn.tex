%---------------------------------------------------------------------
%
%                      resumen.tex
%
%---------------------------------------------------------------------
%
% Contiene el cap�tulo del resumen en ingl�s.
%
% Se crea como un cap�tulo sin numeraci�n.
%
%---------------------------------------------------------------------

\chapter{Abstract}
\cabeceraEspecial{Abstract}

Nowadays, WSNs (Wireless Sensor Networks) are subject to development and deployment constraints, such as the increasing RF spectrum saturation at the unlicensed bands.

\vspace{.1cm}

CNs (Cognitive Networks), leaning on a cooperative communication model, represent a new paradigm aimed at improving spectrum efficiency and wireless communications. CWSNs (Cognitive Wireless Sensor Networks) compound cognitive properties into common WSNs, thereby developing new strategies to mitigate the inefficiency in communications. 

\vspace{.1cm}

It is important to investigate cognitive models to explore their benefit over our WAHSNs, specially constrained in energy and resources. However, few platforms allow their study due to their early research stage, and they still possess scarce or specific features. Investigations take place mainly over simulators, which provide partial and incomplete results.

\vspace{.1cm}

This dissertation presents the development of a versatile platform that brings together cognitive properties into WSNs. Hardware and software models are combined to create an instrument used to investigate CWSNs. The hardware fits WSN requirements in terms of size, cost, and
energy. It allows communication over three different RF bands, and it thus the only cognitive platform for WSNs with this capability.
In addition, its modular and scalable design is widely adaptable to almost any WAHSN application.

\vspace{.5cm}

\begin{table}[h!]
\Large
\scalebox{0.8}{
\begin{tabular}{ l l }
\textbf{\emph{KEY WORDS}}:	& \emph{cognitive networks}, \emph{cognitive radio}, \emph{wireless sensor network}, \emph{platform}, \\ 					&	\emph{testbed}, \emph{cognitive wireless sensor network}, \emph{node}, \emph{device}.
\end{tabular}}
\end{table}

\endinput
% Variable local para emacs, para  que encuentre el fichero maestro de
% compilaci�n y funcionen mejor algunas teclas r�pidas de AucTeX
%%%
%%% Local Variables:
%%% mode: latex
%%% TeX-master: "../Tesis.tex"
%%% End:
