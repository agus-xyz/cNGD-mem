%---------------------------------------------------------------------
%
%                          Cap�tulo 1
%
%---------------------------------------------------------------------

\chapter{Introduction}

\begin{FraseCelebre}
\begin{Frase}
Something in the air
\end{Frase}
%\begin{Fuente}
%fuente
%\end{Fuente}
\end{FraseCelebre}

\begin{resumen}
In this chapter, a global description of the project is shown. This global description covers an introduction to the context,
the need of this project and a brief description if it. It defines goals and the project organization in time. Besides, it is 
described the structure of this dissertation.  
\end{resumen}


%-------------------------------------------------------------------
\section{Background}
%-------------------------------------------------------------------
\label{cap1:sec:background}
%-------------------------------------------------------------------

Ubiquitous computing, with a 40-fold increase between 2010 and 2015 according to the CISCO report \cite{cisco}, 
is nowadays one of the most important trends. Related applications include security and surveillance, industrial control,
infrastructure maintenance, automatic environmental monitoring, domotics, localization and tracking or health. 
\ac{WSNS} provide a technological solution to these challenges, so their growth is closely linked to these fields. 

Together with this significant growth, in the last years, it was a huge and yet increasing wireless networks and sevices deployment caused by the 
great demand by users. This fact, in addition to obvious benefits for the public, brought a remarkable radio spectrum saturation.
This increasing demand for wireless communication presents a challenge to efficient spectrum utilization and spectral 
coexistence.

Regarding spectrum scarcity, most WSN solutions operate on unlicensed frequency bands. Normally they use \ac{ISM} bands,
like the 2.4 GHz band, also used by Wi-Fi or IEEE 802.15.4 devices. For this reason, the unlicensed spectrum bands are 
becoming overcrowded.

The spectrum saturation is reflected over the increasing interferences and other damages between systems. These troubles find their
origin in the increasing amount of traffic, the heterogeneous networks and devices, as well as the inefficient use of the spectrum
they do. Nowadays, techniques focused on expand spectrum efficiency over communications seems totally exploited and new long-term
solutions are required. 

To address this challenge, new techniques such as \ac{CR} arise. \ac{CR} was introduced by Mitola in 1999 \cite{mitola1}
and 2000\cite{mitola2} and it integrates into communication networks concepts as self-learning, self-management, context 
analysis or data sharing. Basically, \ac{CR} enables opportunistic access to the spectrum through cooperation and context
awareness. Spectrum sensing capabilities and cooperation among devices allow for better spectrum utilization and \ac{QOS}.
The concept of \ac{CN} describes a \ac{WN} aware of its environment and able to adapt its communication parameters,
thanks to cooperation, in order to achieve more reliable and efficient communications.

\ac{WSNS} are shown as one of the areas with the highest demand for cognitive networking since the imposed constraints
and hard operation conditions require efficient wireless communications, collaboration among nodes and self-adaptation
against possible environmental incidents. However, there are also some challenges to face \cite{challenges}, specifically the study
of the sensing state, collaboration among devices or decision taking.  

Despite the potential of \ac{CWSN}, they are not deeply explored yet. Real or simulated scenarios scarcely exist and 
realistic platforms are crucial to the improvement and development of this new field. The shortage of \ac{CWSN} devices
or test-beds contributes to the scarcity of results in this area. 

The first \ac{CWSN} node platform was developed in 2011\cite{fcd} by researchers at the laboratory where this project is framed, the
\ac{LSI}, belonging to the \ac{DIE} within the \ac{ETSIT} at \ac{UPM}. This laboratory hosts researching lines related to security, \ac{WSN},
and wireless devices. The developed device, the \ac{FCD}, become the first real generic platform to study \ac{CWSNS} but it was just a
first approach and it was still far to be a stable commercial testbed platform. The node did not fully satisfy requirements in terms of low power consumption, cost and size, and communication capabilities. 

Continuing with the development of the \ac{FCD}, some software implementation were carried out at \ac{LSI}. A combined protocol stack for
three different transceivers including a \ac{HAL}, described in \cite{juanpfc}, was created. This firmware offers
to the application layers ease of use facing communication taks and it brings memory and resources savings compared with
the previous three stacks based-on protocol scheme. As well, in order to deal with cognitive strategy and agorithmics, a software module
that runs in parallel with the application layer was launched. This module, called \ac{CRMODULE} include the theoretical model described
in \cite{conbrok} and it is a crucial achievement for our goals.  


This dissertation describes the development and operation of the \ac{CNGD}, a node platform for \ac{CWSN} development. It includes features and capabilities not found on current devices. \ac{CNGD} must enable and promote \ac{CWSNS} allowing test strategies and better investigation deployments.
The final implementation must integrate the developed software modules \ac{HAL} and \ac{CRMODULE} together with a demo application layer that shows the full operation of the device. 
   
When designing CWSN devices, the fact that WSN nodes are very limited in terms of memory, computational power or energy consumption is crucial and it must be taken into account in the design process. The aim is a flexible scheme based on three different RF bands. On the one hand, as a \ac{WSN} platform, it must have a downward trend regarding power consumption, cost, size and resources. On the other hand, as development platform, the software and hardware architecture must provide advantages when implementing cognitive strategies as a valuable tool. 
 
The design must be modular, so it will not need to be redone entirely with every particular change, and scalable, allowing flexibility in the complexity of applications.

%-------------------------------------------------------------------
\section{Goals}
%-------------------------------------------------------------------
\label{cap1:sec:goals}
%-------------------------------------------------------------------

The project main goal is to design and implement, regarding strong 
efficiency, modularity and scalability constraints, a node for a 
\ac{CWSN}. Develop a demo application layer and integrate it together with
the already implemented \ac{HAL} and \ac{CRMODULE}.

A downtrend regarding cost and power consumption is required, making the device suitable
for commercial purposes. Besides, it must allow the evaluation
of different CN-related strategies.  

It is important to make clear that the design must fill the gaps
and deficiencies that current devices show and provide new desirable capabilities.
Furthermore, a definition of new needs and a researching team sharing,
together with an important review of the \ac{FCD} is needed. 

Here the main goal is fragmented into shorter subgoals:

\begin{itemize}

\item \emph{Previous conclusions and results analysis}.
Study of already tested aspects. Definition of new tests 
to provide new data of interest. These data will help
to take decisions about transceivers, microcontroller,
power supply...

\item \emph{Definition of the node}. Study of needs and constraints 
(cost, resources, size, consumption...) according to the future running applications.
Definition of features and requirements. Definition of the demo application.

\item \emph{Hardware design}. Pursuit of deficiencies over 
the \ac{FCD} design and possible beneficial potentialities 
to achieve the new defined requirements. Design optimization.
Components selection and final layout development. 

\item \emph{Hardware debugging and implementation}.
Component disposition and soldering. PCB design validation.

\item \emph{Application layer development}.
Adaptation to the \ac{HAL} and \ac{CRMODULE}.
Development of the demo application layer integrating
the two already named software modules enabling a 
full test of the platform. The application will include
tipical \ac{WSN} functionalities.

\item \emph{General purpose tests}. 
Tests oriented to prove the proper operation and verify
the achievement of the requirements and constraints compliance.

\item \emph{Results analysis and conclusions}.
Evaluation of capabilities. Study of weaknesses or 
posible improvements. Definition of further studies.

\end{itemize}

%-------------------------------------------------------------------
\section{Project Organization}
%-------------------------------------------------------------------
\label{cap1:sec:projectOrganization}
%-------------------------------------------------------------------

The project organization is here described:

\begin{itemize}

\item \emph{Art state review}.

At this first stage, \ac{CWSNS} related information and specific knowledge
was acquired. Also, it was an approach to the development tools. Goals:  

\begin{itemize}
\item \ac{SN} and \ac{CR} art state review. Commercial solutions review.

\item Adaptation to the developement platform as well as other tools (
documentation, manuals, hardware at the lab...).

\item Absorption of related projects results, either completed or under
development.
\end{itemize}

\item \emph{Hardware design and implementation}.

Over this phase covered from the first definitions and specifications
to the system mounting and implementation. Goals:
tivos:

\begin{itemize}
\item Requirements definition and general node specifications. Constraints and
design criteria.

\item Wireless interfaces and other used devices evaluation.

\item Design optimizations repect to the \ac{FCD}.

\item Hardware full design. Componentes election. Design adjustments and previous
work verification. Layout design.

\item Platform implementation.

\end{itemize}

\item \emph{Software design}

Once the hardware was operating, the work turned over the software. Goals:

\begin{itemize}

\item Adaptation to the developed \ac{HAL} and \ac{CRMODULE}.

\item Software implementation of required functions and application. Source code debugging.

\item Generation of documentation about the developed modules.

\end{itemize}

\item \emph{Tests and evaluation}. Finally the proper operation of the device was 
evaluated and its results analysed.

\begin{itemize}
\item Fully integrated hardware and software test. Spectrum sensing, power consumption and
autonomy, communication among nodes, connectivity to other devices, protocol stacks, 
cognitive capabilities...
 
\item Results interpretation and conclusions review. Statement of further studies 
and future development lines. Found problems evaluation.
ados.
\end{itemize}
\end{itemize}
%-------------------------------------------------------------------
\section{Outline}
%-------------------------------------------------------------------
\label{cap1:sec:outline}
%-------------------------------------------------------------------
% Variable local para emacs, para  que encuentre el fichero maestro de
% compilaci�n y funcionen mejor algunas teclas r�pidas de AucTeX
%%%
%%% Local Variables:
%%% mode: latex
%%% TeX-master: "../Tesis.tex"
%%% End:
