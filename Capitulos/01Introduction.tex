%---------------------------------------------------------------------
%
%                          Cap�tulo 1
%
%---------------------------------------------------------------------

\chapter{Introduction}

\begin{FraseCelebre}
\begin{Frase}
Something in the air
\end{Frase}
%\begin{Fuente}
%fuente
%\end{Fuente}
\end{FraseCelebre}

\begin{resumen}
Resumen
\end{resumen}


%-------------------------------------------------------------------
\section{Background}
%-------------------------------------------------------------------
\label{cap1:sec:background}
%-------------------------------------------------------------------

<<<<<<< HEAD
Ey tu! \ac{WSN}
\ac{CR}
=======
En los �ltimos a�os se ha vivido un enorme, y a�n creciente, despliegue de redes y servicios inal�mbricos originado por la gran demanda y aceptaci�n de los usuarios. Este hecho, adem�s de grandes bondades para el p�blico, ha tra�do consigo una notable saturaci�n del espectro radioel�ctrico que se ha visto reflejado en un  aumento en las interferencias y otros perjuicios entre sistemas. Estos problemas se originan debido a la demanda creciente de tr�fico, la heterogeneidad de redes y terminales existentes, as� como al uso ineficiente que estos hacen del espectro. A d�a de hoy, las t�cnicas centradas en aumentar la eficiencia espectral en las comunicaciones se encuentran pr�cticamente agotadas y se requieren soluciones consistentes a largo plazo.


>>>>>>> a1611710281f8337978a3f73aa4341be414ec914
%-------------------------------------------------------------------
\section{Project Organization}
%-------------------------------------------------------------------
\label{cap1:sec:projectOrganization}
%-------------------------------------------------------------------
a�kldjfa�dsl \ac{WSN},
%-------------------------------------------------------------------
\section{Disposition}
%-------------------------------------------------------------------
\label{cap1:sec:disposition}
%-------------------------------------------------------------------
% Variable local para emacs, para  que encuentre el fichero maestro de
% compilaci�n y funcionen mejor algunas teclas r�pidas de AucTeX
%%%
%%% Local Variables:
%%% mode: latex
%%% TeX-master: "../Tesis.tex"
%%% End:
