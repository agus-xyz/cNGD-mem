%---------------------------------------------------------------------
%
%                          Cap�tulo 3
%
%---------------------------------------------------------------------

\chapter{Review Study}

\begin{FraseCelebre}
\begin{Frase}
Work It Harder Make It Better\\
Do It Faster, Makes Us stronger\\
More Than Ever Hour After\\
Our Work Is Never Over 
\end{Frase}
%\begin{Fuente}
%Fuente
%\end{Fuente}
\end{FraseCelebre}

\begin{resumen}
In this chapter it is defined a model for our cognitive node. Firstly, the First Cognitive Device is evaluated
to obtain some guidelines and potential improvements in our design. The final device implementation requirements and
constraints are defined. Finally, it is carried out a discussion and evaluation about different modules, exposing different 
commercial options and coming into final decissions.
\end{resumen}


%-------------------------------------------------------------------
\section{Cognitive Wireless Sensor Network Node Model}
%-------------------------------------------------------------------
\label{cap3:sec:nodeModel}
%-------------------------------------------------------------------

%-------------------------------------------------------------------
\section{First Cognitive Device Evaluation}
%-------------------------------------------------------------------
\label{cap3:sec:fcdEvaluation}
%-------------------------------------------------------------------

%-------------------------------------------------------------------
\section{System Requirements}
%-------------------------------------------------------------------
\label{cap3:sec:systemRequirements}
%-------------------------------------------------------------------

%-------------------------------------------------------------------
\section{Technology}
%-------------------------------------------------------------------
\label{cap3:sec:technology}
%-------------------------------------------------------------------

%-------------------------------------------------------------------
\subsection{Microcontroller}
%-------------------------------------------------------------------

%-------------------------------------------------------------------
\subsection{Radio Interfaces}
%-------------------------------------------------------------------

%-------------------------------------------------------------------
\subsection{Serial Communication}
%-------------------------------------------------------------------

%-------------------------------------------------------------------
\subsection{Power Supply System}
%-------------------------------------------------------------------

%-------------------------------------------------------------------
\subsubsection{Consumption}
%-------------------------------------------------------------------

%-------------------------------------------------------------------
\subsection{Timing}
%-------------------------------------------------------------------

%-------------------------------------------------------------------
\subsection{Application Capabilities}
%-------------------------------------------------------------------

%-------------------------------------------------------------------
\section{Conclusions}
%-------------------------------------------------------------------
\label{cap3:sec:conclusions}
%-------------------------------------------------------------------

%-------------------------------------------------------------------
%\section*{\NotasBibliograficas}
%-------------------------------------------------------------------
%\TocNotasBibliograficas
%Citamos algo para que aparezca en la bibliograf�a\ldots
%\citep{ldesc2e}

%\medskip

%Y tambi�n ponemos el acr�nimo \ac{CVS} para que no cruja.

%Ten en cuenta que si no quieres acr�nimos (o no quieres que te falle la compilaci�n en ``release'' mientras no tengas ninguno) basta con que no definas la constante \verb+\acronimosEnRelease+ (en \texttt{config.tex}).


%-------------------------------------------------------------------
%\section*{\ProximoCapitulo}
%-------------------------------------------------------------------
%\TocProximoCapitulo

%...

% Variable local para emacs, para  que encuentre el fichero maestro de
% compilaci�n y funcionen mejor algunas teclas r�pidas de AucTeX
%%%
%%% Local Variables:
%%% mode: latex
%%% TeX-master: "../Tesis.tex"
%%% End:
