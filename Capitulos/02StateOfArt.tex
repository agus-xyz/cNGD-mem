%---------------------------------------------------------------------
%
%                          Cap�tulo 2
%
%---------------------------------------------------------------------

\chapter{Cognitive Wireless Sensor Networks: State of Art}

\begin{FraseCelebre}
\begin{Frase}
There's not much I know about you\\
Fear will always make you blind\\
But the answer is in clear view\\
It's amazing what you'll find face to face
\end{Frase}
%\begin{Fuente}
%Fuente
%\end{Fuente}
\end{FraseCelebre}

\begin{resumen}
This chapter shows an approach to the fundamentals of Wireless Sensor Networks and
the current development state of new paradigms such as Cognitive Radio, Cognitive Networks or Cognitive 
Wireless Sensor Networks. It describes current related applications and implementations, and introduce terms used all along
this dissertation.
\end{resumen}


%-------------------------------------------------------------------
\section{Cognitive Radio}
%-------------------------------------------------------------------
\label{cap2:sec:cognitiveRadio}
%-------------------------------------------------------------------

Nowadays, it is widely accepted that the main limitation in next-generation wireless systems is bandwidth scarcity. 
It is commonly believed that there is a crisis of spectrum availability for wireless communications\cite{corvus}.
However, according to regulatory bodies as the \ac{FCC}\cite{fcc} or the \ac{OFCOM}, most radio frequency spectrum is under-utilized while some spectrum bands are heavily used. Military, amateur radio or satellital frecuencies, for instance, are insuffiently utilized compared to cellular networks or the overcrowded \ac{ISM} bands\cite{wpanIssues}.

Most of the spectrum is allocated to specific applications and the static assignment of the spectrum 
results in an inefficient use of it. Figure~\ref{fig:spectrum} shows how the actual utilization in the 3-4 GHz
frequency band is 0.5\% and drops to 0.3\% in the 4-5 GHz band. This seems totally in contradiction to the concern of spectrum
shortage.

Spectrum utilization depends strongly on time and place, however, fixed spectrum allocation prevents specific assigned frequencies 
from being used, even when this use would not cause noticeable interference to the assigned service. These facts lead to the current inneficiency situation where the utilization of the total spectrum can be considered around 10\% and more than 95\% of the use is below 3GHz.

%datos triples sacaos de http://www.cmpe.boun.edu.tr/WiCo/doku.php?id=research#cognitive_radio
%WiMax Networks & Cognitive Radio, namely WiCo, is a research group functioning under Computer Networks Research Laboratory (NetLab) of Computer Engineering Department at bla 

\figuraEx{Vectorial/Todo}{width=.5\textwidth}{fig:spectrum}%
{A snapshot of the spectrum utilization up to 6 GHz in an urban area: taken at
mid-day with 20 kHz resolution taken over a time span of 50 microseconds with a 30
degree directional antenna at Berkeley Wireless Research Center \cite{seung}.}{A snapshot of the spectrum utilization up to 6 GHz in an urban area at BWRC}

The concept of \ac{CR} was first published by Joseph Mitola III and Gerald Q. Maguire, Jr. in 1999\cite{mitola1} and later on within Mitola's phD Dissertation in 2000\cite{mitola2}. It describes a novel paradigm for wireless communication in which a wireless device changes its transmission or reception parameters to communicate efficiently. This alteration of parameters is based on the active monitoring of several factors in the external and internal radio environment, such as radio frequency spectrum, user behavior, and network state. The idea was thought of as an ideal goal towards which a \ac{SDR} platform should evolve.

The cognitive radio is defined as an intelligent wireless communication system that is aware of its environment and uses the methodology of under-
standing-by-building to learn from the environment and adapt to statistical variations in the input stimuli, with two primary objectives:
\begin{itemize}
\item Hihighly reliable communications.
\item Efficient utilization of the radio spectrum.
\end{itemize}

Although the concept of \ac{CR} was defined originally as an extension to \ac{SDR}\cite{mitola1}, which is able to reason about
external factors, recently the term is mostly used in a narrower sense. \ac{FCC} suggests in~\cite{fcc1} that any radio having the adaptive spectrum awareness should be referred to as \ac{CR}:
\begin{quote}

\emph{
``A cognitive radio (CR) is a radio that can change its transmitter parameters based on
interaction with the environment in which is operates. The majority of cognitive radios will
probably be SDRs (Software Defined Radios), but neither having software nor being field
programmable are requirements of a cognitive radio.'' 
}

\end{quote}
Attending to some subtle differences between systems we can differenciate two main different types of \ac{CR}:

\begin{itemize}
\item \emph{Spectrum-Sensing Cognitive Radio}, in which only the radio-frequency spectrum is considered.\cite{brainempower}
\item \emph{Full Cognitive Radio  or Mitola radio}, in which every possible parameter observable by a wireless node is considered.\cite{mitola1}
\end{itemize}    

Although cognitive radio was initially thought of as full cognitive radio, most research work focuses on spectrum-sensing cognitive radio,  particularly in the TV bands. The great problem in spectrum-sensing cognitive radio is designing high-quality spectrum-sensing devices and algorithms for exchanging the so called knoledge domain \cite{brainempower}. The practical implementation of spectrum-management functions is a complex and multifaceted issue, since it must address a variety of technical and legal requirements.

Picture \ref{fig:CWSNsoftModel} ilustrates the changes the \ac{OSI} model suffer when affected by cognition. \ac{CR} model can be referred to the first and second layers, thus the \emph{Phisical} and \emph{Link} layers. 

\figura{Vectorial/Todo}{width=.5\textwidth}{fig:CWSNsoftModel}%
{CWSN protocol model}
%poner una unica pila


The main functions of \ac{CR} devices are:\cite{dsa}\cite{CRfunctionality}

\begin{itemize}

\item \emph{Spectrum sensing}: An important requirement is detecting unused spectrum and sharing it, without causing interferences to other users; Spectrum-sensing techniques may be grouped into three categories:

\begin{itemize}
\item \emph{Transmitter detection}: \ac{CR} must have the capability to determine if a signal from a primary transmitter is locally present in a certain spectrum. Enclosed here we can find approaches such as \emph{matched filter detection, energy detection or cyclostationary-feature detection} are common.

\item \emph{Cooperative detection}: Refers to spectrum-sensing methods where information from multiple \ac{CR} users is integrated\cite{sensing}.

\item \emph{Interference-based detection}. This technique is not so commonly used. 
\end{itemize}

\item \emph{Power Control}: Power control is used for both opportunistic spectrum access and spectrum sharing CR systems for finding the cut-off level in SNR supporting the channel allocation and imposing interference power constraints for the primary user's protection respectively. In~\cite{threshold} a joint power control and spectrum sensing is proposed for capacity maximization.

\item \emph{Spectrum management}: Capturing the best available spectrum to meet user communication requirements, while not creating undue interference to other users. \ac{CR} should decide on the best spectrum band (over the available range) to meet \ac{QOS} requirements; therefore, spectrum-management functions are required for \ac{CR}. Spectrum-management functions are classified as \emph{Spectrum analysis} and \emph{
Spectrum decision}.

\end{itemize}

Realizing that CR technology has the potential to exploit the inefficiently utilized licensed bands without causing interference to incumbent users, the FCC released a Notice of Proposed Rule Making which would allow unlicensed radios to operate in the TV-broadcast bands. The IEEE 802.22 working group, formed in November 2004, is tasked with defining the air-interface standard for wireless regional area networks (based on CR sensing) for the operation of unlicensed devices in the spectrum allocated to TV service \cite{80222}.

%-------------------------------------------------------------------
\section{Cognitive Networks and Cognitive Radio Networks}
%-------------------------------------------------------------------
\label{cap2:sec:cognitiveNetworks}
%-------------------------------------------------------------------

On the last years, ``cognitive'' or ``smart'' have become \emph{trending topics} being applied to many fields, included
to communication technologies. Having a look into the 90s literature, easily at least we find mentions about
smart antennas\cite{smartantenna}, smart radios\cite{smartradio}, smart packets\cite{smartpackets}, \ac{CR}\cite{mitola1}\cite{brainempower}, cognitive packets\cite{cognitivepackets} and \ac{CN}\cite{cn1}\cite{cn2}.
Nevertheless, there does not seem to exist a commonly accepted definition of what these terms mean when applied to 
networking technologies.

The concept of \ac{CN} has been hanging out the collective psyche of the networking and wireless researching field for long. 
The first approach was made by Mitola\cite{mitola1} when briefly describes how the \ac{CR} could interact within the system-level scope 
of a \ac{CN}. Saracco\cite{saracco} talks about \ac{CN}s in his investigation into the future information technology. Mahonen et al. \cite{cn1}
 discuss \ac{CN}s with respect to future mobile \ac{IP} networks. None of these previous references, however, express clearly what 
a \ac{CN} is, how it should work and which problems it should solve.

The role that \ac{CR} had in inspiring the formulation of \ac{CN} concept, made, in some cases, \ac{CN}s being described as networks of 
\ac{CR}s\cite{mitola1}\cite{neel}. Recent research can be divided into two categories: \ac{CRN} and \ac{CN}s. 

For \ac{CRN}, Mitola mentions how \ac{CR}s could interact within the system-level scope of a \ac{CN}\cite{mitola2}. Neel\cite{gamecr}
and Haykin\cite{brainempower} continue this line of thinking, examinating multi-user networks of \ac{CR}s as a game.
The scope of \ac{CRN}s still remains primarily on \ac{MAC} and \ac{PHY} layers, but now operating with some end-to-end objective. 
In a \ac{CRN}, the individual radios take most of the cognitive decisions, although they may act in cooperation. Some suggested
applications for \ac{CRN}s include cooperative spectrum sensing\cite{coopsensing}\cite{coopsensing1} and emergency radio
networks \cite{cremergency}. Raychaudhuri presents in\cite{cnrarchitecture} a general architecture for \ac{CRN}s.

Regarding \ac{CN}s, Clark proposes, in which was perhaps the first mention of \ac{CN} rather than \ac{CRN},
a network that can

\begin{quotation}
``assemble itself given high level instructions, reassemble itself as requirements change, automatically discover when 
something goes wrong, and automatically fix a detected problem or explain why it cannot do so.''
\end{quotation}

This would be achieved with the use of a \ac{KP} that transcends layers and domains to make global 
cognitive decisions. The \ac{KP} will add intelligence and weight to the edges of the network, and context sensitivity 
to its core. Saracco stated\cite{saracco} that the change from network intelligence controlling resources to having context 
sensitivity will help "flatten" the network by moving network intelligence into the core and control further out to the 
edges of the network. \ac{CRN}s differ from \ac{CN}s in that their action space extends beyond the \ac{MAC} and \ac{PHY} layers and the network may 
consist of more than just wireless devices. Furthermore, CN nodes may be less autonomous than a CRN node.


First full definition of \ac{CN} was postulated by Thomas\cite{thomas} in which was his PhD Dissertation. Hi proposed
the idea of a CN:

\begin{quotation}
\emph{
a network composed of elements that, through learning and reasoning, dynamically adapt to varying network conditions in 
order to optimize end-to-end performance. In a CN, decisions are made to meet the requirements of the
network as a whole, rather than the individual network components. 
}
\end{quotation}


The adaptations that are performed over usual networks are commonly reactive, taking place after a problem has
occurred. Thomas advanced a paradigm that had the promise to remove these limitations by allowing networks to observe, act, and 
learn in order to optimize their performance. \ac{CN}s description in \cite{thomas} talks about intelligently select and adapt radio spectrum, transmission power, antenna parameters and routing tables. By formalizing the design, architecture and tradeoffs of cognition at the network level, Thomas work had a broad impact in advancing the paradigm of intelligent communication devices.

%-------------------------------------------------------------------
\subsection{Example}
%-------------------------------------------------------------------
This example was published \cite{thomas} in and it is inspired and influenced by Daniel Friend's example in \cite{example}. It illustrate the need for end-to-end rather than link adaptations.


\figuraEx{Bitmap/Capitulo2/example}{width=.5\textwidth}{fig:example}%
{Simple relay network for a wireless network. Vertices represent wireless connectivity}{Simple relay network for a wireless network.}


Consider an ad-hoc data session between a source node $S_{1}$ and a destination node $D_{1}$ as shown in Figure~\ref{fig:example}. The
source node does not have enough power to reach $D_{1}$ directly, so it must route traffic
through intermediate nodes $R_{1}$ and-or $R_{2}$. Assume that the end-to-end goal is to have the
highest probability of successful transmission. The routing layer will determine routes based
on minimum hop count which, in this case, includes either $R_{1}$ or $R_{2}$. Node $S_{1}$ will make a
link-layer adaptation, selecting between $R_{1}$ and $R_{2}$ based on their \ac{SINR}. From the standpoint of the 
link layer in node $S_{1}$, this ratio correlates with the probability that the transmitted packets will arrive correctly at the relay node.
However, without additional information, this selection does not guarantee anything about the end-to-end packet delivery probability from $S_{1}$ to $D_{1}$.
In contrast to a link adaptation, the \ac{CN} might use some combination of observations from
all nodes to compute the total path outage probabilities from $S_{1}$ to $D_{1}$ through $R_{1}$ and $R_{2}$.
This shows the benefit of an end-to-end scope, but there is another advantage to the
\ac{CN}, its cognitive capability. To illustrate this, we modify the original scenario to include
both $S_{1}$ and $S_{2}$ as source nodes, both routing traffic through $R_{2}$.
Suppose that the learning mechanism measures outages by determining the fraction of packets successfully delivered
from the source to its destination.

If $R_{2}$ becomes congested because of a large volume of traffic coming from $S_{2}$ this becomes
apparent to the cognitive process by the lack of successful packet delivery statistics provided
to $S_{1}$ and $S_{2}$. The learning mechanism recognizes that the system has changed and that
routing through $R_{2}$ is not optimal. The cognitive process then directs the traffic toward
another route. The \ac{CN} does not explicitly know that there is congestion at node $R_{2}$ because
this information is not included in the \ac{SINR} observations. Nevertheless, it is able to infer
from the reduced throughput that there may be a problem. It is then able to respond
to the congestion, perhaps by routing traffic through $R_{1}$ and/or $R_{3}$. This example shows
the power of \ac{CN}s in optimizing end-to-end performance as well as reacting to unforeseen
circumstances.
 
%-------------------------------------------------------------------
\section{Wireless Sensor Networks}
%-------------------------------------------------------------------
\label{cap2:sec:wirelessSensorNetworks}
%-------------------------------------------------------------------

A \ac{WSN} consists of spatially distributed autonomous ``nodes'' not relying on a pre existing communication infrastructure,  to monitor physical or environmental conditions. Number of nodes vary from a few to several hundreds or even thousands, where each node is connected to one (or sometimes several) sensors. Nodes are generally simple and low-resources embedded system with high cost and size constraints. These constraints result in corresponding constraints on resources such as energy, memory, computational speed and communications bandwidth. Usual architecture of a node is
divided in:

\begin{itemize}
\item \emph{Sensing subsystem}. Responsible for sensing physical parameters of the environment. Typical monitored parameters are temperature,
	sound, light, humidity, vibrations, pressure, movement, presence, body registers...
\item \emph{Computational subsystem}. It proccess the information obtained by the sensors and process it. It controls all general operations of the 		node, and runs the desired application.  
\item \emph{Communication subsystem}. To carry through all the messages transmission and reception with neighbor nodes. Main goal is to make
	the information gets to some destination, usually a gateway or storaging node.
\end{itemize}

\figura{Vectorial/Todo}{width=.5\textwidth}{fig:WSNnodeModel}%
{WSN node model}

The development of wireless sensor networks was motivated by military applications such as battlefield surveillance; today such networks are used in a wide range of applications, considered as the main technology to develop intelligent ambiences:

\begin{itemize}
 \item	\emph{Industrial Monitoring}. \ac{WSN} are applied mainly for machine health monitoring, water quality or management, and industral sense
         and control applications avoiding wire deployment.

 \item	\emph{Surveillance}. Distributed sensors over infrastructure as bridges, tunnels or buildings help to collect data to prevent
	damages or problems derivated from load excesses, weather, vibrations or stress. 

 \item	\emph{e-Health}. Not deeply explored yet. It presents new uses for \ac{WSN} to sense body parameters and observate behavioral patterns.
	These networks are used to detect or prevent occupational and home accidents, to improve diagnoses, monitoring sick people and other
	medical uses.

 \item	\emph{Environmental monitoring}. Applications regarding this fiel are diverse. Air quality monitoring, air pollution monitoring, forest fire 		detection, landslide detection, water quality monitoring or natural disaster prevention are some of the common uses.

 \item	\emph{Smart home monitoring}. Monitoring the activities performed in a smart home is achieved using wireless sensors embedded within everyday 		objects. State changes to objects based on human manipulation is captured by the \ac{WSN} enabling activity-support services.

 \item  \emph{Passive localization and tracking}. Applications oriented to detect where something took place or track presences over an area.

 \item  \emph{Agriculture}. Commonly used on greenhouses where irrigation management and ambient control is essential for a proper accurate
	agriculture.
\end{itemize}

Some of the main properties affecting \ac{WSN} are: 

\begin{itemize}
\item \emph{Dinamic topology}. In a \ac{WSN} it is common to suffer drops in the number of nodes or changes over the environment that
	affects the topology, nodes must be able to adapt themselves to new topologies to enable operative communications. On the same way,
	topologies must be scalables since a network might have tens of nodes or hundred of them.

\item \emph{Autonomous operation}. There is not need any infrastructure for a \ac{WSN} to operate. Its nodes act as information transmitters, 		receivers or routers. However, it usually exists a gateway which gather all the information over the network and pass it to another 		device. 

\item \emph{Multihop or broadcast communications}. It is common the use of some protocol to enable multihop messaging. Nevertheless, broadcasting
	 is also very expanded.

\item \emph{Power consumption}. One of the most important factors. Using a very constrained amount of energy, devices must achieve a tradeoff
	between autonomy and troughput. A \ac{WSN} node must meet a low-consumption microcontroller, as well as radio interfaces and software 
	equally featured.

\item \emph{Hardware constraints}. In order to achieve a low-power consumption, it is essential for the hardware to be as most straightforward
	as possible, coming into a limited computing capability.

\item \emph{Production costs}. Since \ac{WSN}s nature implies having a high number of nodes to be trustable, production of large amounts of them must 		provide a cheap unitary price.

\end{itemize}


Most important communication techonologies and protocols for \ac{WSN} are based on the IEEE 802.15.4 \cite{wpanieee} standard for \ac{WPAN}s.
The standard goal is a low-power communication among nearby devices without underlying infrastructure. The standard only defines \ac{MAC} and
\ac{PHY} layers of the \ac{OSI} model. Some expanded protocols based on the standard are ZigBee$^{TM}$ *****REF, WirelessHART*****REF, ISA100.11*****REF or MiWi$^{TM}$**********REF specifications. Another popular communication standard is IEEE 802.11***** REF, in which is based
WiFi*****REF. It is a \ac{IP} standard based on the final user and does not meet low-consumption constraints, however, convergence towards
full \ac{IP} has brought new standards such as \ac{6LOWPAN} *****REF, that enables \ac{IP} packeting over 802.15.4 based-on networks.


\begin{figure}[h]
\centering
%
\begin{SubFloat}
{\label{fig:cap2:wsn_router}%
\scriptsize{WSN con utilizaci�n de routers para la conexi�n con el gateway.}}%
\includegraphics[angle=270,width=0.45\textwidth]%
{Imagenes/Vectorial/Capitulo2/wsn_router}%
\end{SubFloat}
\qquad
\begin{SubFloat}
{\label{fig:cap2:wsn_adhoc}%
\scriptsize{WSN implementada como una red ad hoc.}}%
\includegraphics[angle=270,width=0.45\textwidth]%
{Imagenes/Vectorial/Capitulo2/wsn_adhoc}%
\end{SubFloat}
\caption{Posibles implementaciones de una red de sensores inal�mbrica (WSN).}
\label{fig:cap2:wsn}
\end{figure}

%-------------------------------------------------------------------
\section{Cognitive Wireless Sensor Networks}
%-------------------------------------------------------------------
\label{cap2:sec:cognitiveWirelessSensorNetworks} 
%-------------------------------------------------------------------

\ac{CWSN}s arises as a natural evolution of traditional \ac{WSN}s since IEEE standards used in \ac{WSN}s already postulate access to several
bands of the spectrum. Additionally, \ac{QOS} and low-consumption requirements that \ac{WSN}s state, fits the goals of \ac{CR}. Currently, \ac{ISM} bands, where most of \ac{WSN}s place their communications, show a crowded scene where \ac{CR} techniques might significantly help the newtork operation. Applying cognitive capabilities seeks a more autonomous operation mode able to optimize certain parameters in real time. Thus, these techniques, based on distributed intelligence, contribute to improve operation and general devices efficiency.

Basically, \ac{CR} and \ac{CN} techniques are included into \ac{WSN}s. This idea suposses an increment of complexity over the executed algorithms
and overload the control data flow over the network.

In \cite{conbrok} oportunities and trends arising from networks and nodes cooperation are mentioned. It is seen as a chance to improve
general features and dynamic adaptation capabilities. \cite{conbrok} proposes a model based on agents implementing basic functions 
and keeping a \ac{VCC}. This \ac{VCC} takes part on the \ac{KP} that agents use to exchange gestion messages.

*************** LAST REFERENCIAS DE JUAN Y PROBLEMS
   
%-------------------------------------------------------------------
\subsection{Current Devices}
%-------------------------------------------------------------------
\label{cap2:sec:currentDevices}
%-------------------------------------------------------------------

Attending real prototypes for \ac{CWSN} applications, implementations respond to testbed platforms and simulators. However,
very scarce variety of solutions, very poorly featured yet, exists nowadays.

QUe hacen los simuladores
que hacen los testbed


hablar del FCD como unico dispositivo y aun asi no sirve pa na.

simuladores:

testbeds:

standares:

conclusion  

%-------------------------------------------------------------------
%\section*{\NotasBibliograficas}
%-------------------------------------------------------------------
%\TocNotasBibliograficas

%Citamos algo para que aparezca en la bibliograf�a\ldots
%\citep{ldesc2e}

%\medskip

%Y tambi�n ponemos el acr�nimo \ac{CVS} para que no cruja.

%Ten en cuenta que si no quieres acr�nimos (o no quieres que te falle la compilaci�n en ``release'' mientras no tengas ninguno) basta con que no definas la constante \verb+\acronimosEnRelease+ (en \texttt{config.tex}).


%-------------------------------------------------------------------
%\section*{\ProximoCapitulo}
%-------------------------------------------------------------------
%\TocProximoCapitulo

%...

% Variable local para emacs, para  que encuentre el fichero maestro de
% compilaci�n y funcionen mejor algunas teclas r�pidas de AucTeX
%%%
%%% Local Variables:
%%% mode: latex
%%% TeX-master: "../Tesis.tex"
%%% End:
