%---------------------------------------------------------------------
%
%                          Cap�tulo 5
%
%---------------------------------------------------------------------

\chapter{Implementation Study}

\begin{FraseCelebre}
\begin{Frase}
Time of your life
\end{Frase}
\begin{Fuente}
Thomas \& Guy-Manuel, Daft Punk
\end{Fuente}
\end{FraseCelebre}


\begin{resumen}
In this chapter, tasks related to the physical implementation of the platform are described. These tasks cover from the layout design, 3D modelling, and GERBER files generation to the final components mounting. Decisions about components or layout designs, and final results are exposed.  
\end{resumen}



%-------------------------------------------------------------------
\section{Introduction}
%-------------------------------------------------------------------
\label{cap5:sec:introduction}
%------------------------------------------------------------------

%double layer, reduce size
%0603 - measures - tipical consumption
%3d model

%PCBCART - fabricante GERBERS

%-------------------------------------------------------------------
\subsection{Mounting}
%-------------------------------------------------------------------

%soldering station, manuales how to
%pruebas para ir corrigiendo errores

%-------------------------------------------------------------------
\section{Transceivers - $\mu$Trans 434/868}
%-------------------------------------------------------------------
\label{cap5:sec:transceivers}
%------------------------------------------------------------------

%FIGURA 3D

%-------------------------------------------------------------------
\subsection{Description}
%-------------------------------------------------------------------


%minimal spacing, track width, vias under the micro, pad to measure clock, ground plane

%SIZE, FIGURA

%holes to solder. footprint - porque FIGURA

%tabla de componentes con package, units, value,symbol


%-------------------------------------------------------------------
\subsection{Layout}
%-------------------------------------------------------------------
%Referencias a ANEXO

%-------------------------------------------------------------------
\subsection{Testing}
%-------------------------------------------------------------------

%primero probado en FCD, errores detectados
%osciloscopio para sensar varias se�ales
%handshaking
%network detection
%data transmission reception

%-------------------------------------------------------------------
\subsection{Final Result}
%-------------------------------------------------------------------

%FOTOS

%-------------------------------------------------------------------
\section{Main Board - cognitiveNextGenerationDevice}
%-------------------------------------------------------------------
\label{cap5:sec:currentDevices}
%-------------------------------------------------------------------

%-------------------------------------------------------------------
\subsection{Layout}
%-------------------------------------------------------------------

%-------------------------------------------------------------------
\subsection{Mounting}
%-------------------------------------------------------------------

%-------------------------------------------------------------------
\subsection{Testing}
%-------------------------------------------------------------------

%-------------------------------------------------------------------
\section{Serial Communication Board - rs232SHIELD}
%-------------------------------------------------------------------
\label{cap5:sec:rs232shield}
%-------------------------------------------------------------------

%-------------------------------------------------------------------
\subsection{Layout}
%-------------------------------------------------------------------

%-------------------------------------------------------------------
\subsection{Mounting}
%-------------------------------------------------------------------

%-------------------------------------------------------------------
\subsection{Testing}
%-------------------------------------------------------------------

%-------------------------------------------------------------------
\section{Transceivers - chargerSHIELD}
%-------------------------------------------------------------------
\label{cap5:sec:chargerShield}
%-------------------------------------------------------------------

%-------------------------------------------------------------------
\subsection{Layout}
%-------------------------------------------------------------------

%-------------------------------------------------------------------
\subsection{Mounting}
%-------------------------------------------------------------------

%-------------------------------------------------------------------
\subsection{Testing}
%-------------------------------------------------------------------

%-------------------------------------------------------------------
\section{Conclusions}
%-------------------------------------------------------------------
\label{cap5:sec:conclusions}
%-------------------------------------------------------------------


%------------------------------------------------------------------
%\section*{\NotasBibliograficas}
%-------------------------------------------------------------------
%\TocNotasBibliograficas

%Citamos algo para que aparezca en la bibliograf�a\ldots
%\citep{ldesc2e}

%\medskip

%Y tambi�n ponemos el acr�nimo \ac{CVS} para que no cruja.

%Ten en cuenta que si no quieres acr�nimos (o no quieres que te falle la compilaci�n en ``release'' mientras no tengas ninguno) basta con que no definas la constante \verb+\acronimosEnRelease+ (en \texttt{config.tex}).


%-------------------------------------------------------------------
%\section*{\ProximoCapitulo}
%-------------------------------------------------------------------
%\TocProximoCapitulo

%...

% Variable local para emacs, para  que encuentre el fichero maestro de
% compilaci�n y funcionen mejor algunas teclas r�pidas de AucTeX
%%%
%%% Local Variables:
%%% mode: latex
%%% TeX-master: "../Tesis.tex"
%%% End:
