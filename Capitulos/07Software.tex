%---------------------------------------------------------------------
%
%                          Cap�tulo 7
%
%---------------------------------------------------------------------

\chapter{Software}

\begin{FraseCelebre}
\begin{Frase}
Too long, can you feel it? \\
Too long, oh can you feel it ?
\end{Frase}
\begin{Fuente}
Thomas \& Guy-Manuel, Daft Punk
\end{Fuente}
\end{FraseCelebre}

\begin{resumen}
This chapter gives a vision about the main adaptations required for the firmware to be adapted to the platform, new software implementations and behaviour of the final demo application.
\end{resumen}

%-------------------------------------------------------------------
\section{Tools}
%-------------------------------------------------------------------
\label{cap7:sec:tools}
%-------------------------------------------------------------------

%-------------------------------------------------------------------
\subsection{MPLAB X}
%-------------------------------------------------------------------
\label{cap7:sub:mplabx}
%-------------------------------------------------------------------
MPLAB X \ac{IDE} is a multiplatform software program to develop applications for Microchip microcontrollers and digital signal controllers. It is called an Integrated Development Environment  because it provides a single integrated ``environment'' to develop code for embedded microcontrollers.

\figura{Bitmap/Capitulo7/mplabx}{width=.5\textwidth}{fig:cap7:mplabx}%
{MPLAB X IDE logo.}

MPLAB X is based on the open source NetBeans IDE from Oracle. Some of its main features are:

\begin{itemize}
    \item Supports Multiple Configurations within your projects.
    \item Support for multiple Debug Tools of the same type.
    \item Supports Live Parsing.
    \item Supports hyperlinks for fast navigation to declarations and includes.
    \item MPLAB X can Track Changes within your own system using local history.
\end{itemize}



%-------------------------------------------------------------------
\subsection{Programmer: ICD 3}
%-------------------------------------------------------------------
\label{cap7:sub:icd3}
%-------------------------------------------------------------------

\figura{Bitmap/Capitulo4/icd3}{width=.8\textwidth}{fig:cap4:icd3}%
{ICD3 programmer configuration.}

\ac{ICD} 3 is a device belonging to Microchip products that allows to get the microcontroller programmed. In addition, it enables run-time debugging using the \ac{IDE}. Up to 6 breakpoints can be enabled. This feature is quite valuable regarding the platform has a development character. Further information can be consulted at the manual\cite{icd3}.

%-------------------------------------------------------------------
\section{Firmware and Hardware Abstraction Layer}
%-------------------------------------------------------------------
\label{cap7:sec:hal}
%-------------------------------------------------------------------

When adapting the firmware to the developed hardware platform, it was needed to deal with minor changes resulting from its novelty and also because of required hardware adaptations.

Some minor bugs that impeded the right sleeping modes operation or the spectrum sensing were fixed. As well, due to lack of tests when using three \ac{RI}s at once, some software configurations were not properly set. Current firmware version corrects these issues and it is fully operative over the hardware.

Other adaptations taken when developing software aim at stablish better ease of use. An easier configuration scheme, \ac{USB} tracing modes, \ac{RI} power control functions, and new modules definition are the developed functionalities.

Current and older versions of the firmware, along with small software related tests, can be found in a public GitHub\footnote{Web-based hosting service for software development projects that use the Git revision control system} repository: \emph{https://github.com/agus-xyz/cognitiveNGD}. Figure \ref{fig:cap7:repoqr} provides a \ac{QR} code to instantly access the repository.

\figura{Bitmap/Capitulo7/repoqr}{width=.3\textwidth}{fig:cap7:repoqr}%
{cNGD software repository QR access.}

%-------------------------------------------------------------------
\subsection{Platform options}
%-------------------------------------------------------------------
\label{cap7:subsec:usbtraces}
%-------------------------------------------------------------------
When configuring the firmware to operate over different hardware platforms such as the \ac{FCD} expanded version, the \ac{CNGD}, or a manually  configurable platform, several changes at the configuration firmware files were needed.

With the changes that the current firmware implements, adapting the firmware to any of these platforms is done by changing a single option at \emph{Include/HardwareConfig.h} file.

\begin{itemize}
\item \emph{cNGD\_PLATFORM}. Automatically sets a suitable configuration for the \ac{CNGD}. \emph{MIWI\_0434\_RI} is stablished over \emph{MRF49XA\_1} transceiver, \emph{MIWI\_0868\_RI} is built over \emph{MRF49XA\_2}, and \emph{MIWI\_2400\_RI} does it on \emph{MRF24J40} transceiver. Remaining configurations to adequately take this options are automatically changed. 

\item \emph{FCD\_Exp\_PLATFORM}. This possibility enables the configuration to employ the \ac{FCD} adapted to the lately developed expansion board. It adapts the \emph{MIWI\_0434\_RI} over \emph{MRF49XA\_1} transceiver, and \emph{MIWI\_2400\_RI} at \emph{MRF24J40}.

\item \emph{MANUAL\_PLATFORM}. This option gives freedom to the user to full configure the employed hardware. Any of the possible MIWI and WIFI   
\ac{RI}s can be chosen over different transceivers such as \emph{MRF49XA\_X, MRF89XA, MRF24J40, or MRF24WB0M}. Moreover, the used \ac{SPI} for interfacing the \ac{RI} is also configurable together with the external interruption line.

\end{itemize}


Basically, the old macros that used to be changed manually for each platform are now atomatically configured when set on of the previous options. Main changes take place over \ac{RI} configurations and pinouts.

Other configurable option at this very same file is the possibility for debugging traces. This option is controlled enabling or disabling the \emph{ENABLE\_CONSOLE} option. Then, within each platform independent configuration, it is possible to select the module to output traces.

%-------------------------------------------------------------------
\subsection{USB tracing}
%-------------------------------------------------------------------
\label{cap7:subsec:usbtraces}
%-------------------------------------------------------------------

%

%-------------------------------------------------------------------
\subsection{Radio Interfaces Power Control Functions}
%-------------------------------------------------------------------
\label{cap7:subsec:usbtraces}
%-------------------------------------------------------------------
After including power control modules for each \ac{RI} at the \ac{CNGD}, it was needed some software adaptations to fully make them operate. For this, a pair of functions have been included at the firmware \ac{HAL}, \emph{SwitchRIOn(radioInterface ri)} and \emph{SwitchRIOff(radioInterface ri)}.

These functions set to high or low the signal that drives the power switch at each \ac{RI}. Received parameter is the chosen \ac{RI}. Return value might be \emph{NO\_ERROR} or an error code.

%-------------------------------------------------------------------
\subsection{Headers and other new definitions}
%-------------------------------------------------------------------
\label{cap7:subsec:usbtraces}
%-------------------------------------------------------------------
For an easy use and programmation of pins at the headers and useful components as leds or buttons, some global definitions at the firmware provide access to them.

To access pins at the headers, applications can make use of the \emph{HEADERS\_XX} label definition and respective \emph{HEADERS\_XX\_TRIS} to configure the sort of pin. \emph{XX} notation must be replaced with the desired pin number. For instance, in order to access pin 6, the right pin definition would be \emph{HEADERS\_06}.************************

Leds label definitions are \emph{LED1}, \emph{LED2}, and \emph{LED3} respectively. These pins are configured as output when doing node initialization.

%-------------------------------------------------------------------
\section{Demo Application Layer}
%-------------------------------------------------------------------
\label{cap7:sec:applicationLayer}
%-------------------------------------------------------------------

Demo application layer was thought to be showed during these Master thesis presentation. It implements a \ac{WSN} application in which a transmitter and receiver module set communication and exhibit some platform communication capabilities. 




%
% Variable local para emacs, para  que encuentre el fichero maestro de
% compilaci�n y funcionen mejor algunas teclas r�pidas de AucTeX
%%%
%%% Local Variables:
%%% mode: latex
%%% TeX-master: "../Tesis.tex"
%%% End:
